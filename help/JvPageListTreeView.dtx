##Package: PageComps
##Status: Completed,Checked (peter3)
----------------------------------------------------------------------------------------------------
@@JvPageListTreeView.pas
Summary
  Contains the the TJvPageListTreeView and TJvSettingsTreeView components.
<INCLUDE JVCL.UnitText.dtx>
Author
  Peter Th�rnqvist

----------------------------------------------------------------------------------------------------
@@TJvCustomPageListTreeView
Summary
  Base class for treeviews that has a PageList property of type IPageList.
Description
  Use TJvCustomPageListTreeView as a base class when developing treeview components that should have
  the ability to be associated with an IPageList implementation.

  TJvCustomPageListTreeView has the same functionality as a normal treeview but in addition it can also
  automatically change the active page of an IPageList implementation when the selected node changes.
  Internally, the TJvCustomPageListTreeView component creates its tree nodes as TJvPageIndexNodes. A
  TJvPageIndexNode has a PageIndex property that specifies which page in the PageList to active when
  the node is selected.

  Since the nodes in the treeview is surfaced as normal TTreenodes, you must typecast the TTreeNode to
  TJvPageIndexNode in code to access the PageIndex property.

  To put a PageList enabled treeview on a form, don't create instances of TJvCustomPageListTreeView.
  Use the TJvPageListTreeView component instead. TJvPageListTreeView only publishes the properties of
  TJvCustomPageListTreeView without adding any new functionality.

----------------------------------------------------------------------------------------------------
@@TJvCustomPageListTreeView.Items
Summary
  Lists the individual nodes that appear in the tree view control.
Description
  Individual nodes in this tree view are TJvPageIndexNode objects but they are surfaced as standard
  TTreeNode objects. To access the additional properties and methods of the TJvPageIndexNode class, you
  must type cast the node to a TJvPageIndexNode. The individual nodes can be accessed by using the
  Items property along with the item's index into the tree view. For example, to access the second item
  in the tree view, you could use the following code.

  <AUTOLINK OFF>
  <CODE>
    (Delphi)
      MyTreeNode := TreeView1.Items[1];
    (C++Builder)
      MyTreeNode = TreeView1->Items[1];
  </CODE>
  <AUTOLINK ON>

  When setting this property at design-time in the Object Inspector the Tree View Items Editor appears.
  Use the New Item and New SubItem buttons to add items to the tree view. Use the Text property to
  modify what text is displayed in the label of the item.

  At run-time nodes can be added and inserted by using the TTreeNodes methods AddChildFirst, AddChild,
  AddChildObjectFirst, AddChildObject, AddFirst, Add, AddObjectFirst, AddObject, and Insert.
Note
  Accessing tree view items by index can be time-intensive, particularly when the tree view contains
  many items. For optimal performance, try to design your application so that it has as few
  dependencies on the tree view�s item index as possible.
See Also
  AddChildFirst, AddChild, AddChildObjectFirst, AddChildObject, AddFirst, Add, AddObjectFirst,
  AddObject, Insert

----------------------------------------------------------------------------------------------------
@@TJvCustomPageListTreeView.PageDefault
Summary
  Specifies the index of the default page.
Description
  Use DefaultPage to specify the value of the PageIndex property for new nodes. For example, settings
  this property to 0 will set the PageIndex of new nodes to 0, setting it to 1 will set the Pageindex
  to 1, etc.
See Also
  TJvCustomPageListTreeView.PageLinks, TJvCustomPageListTreeView.PageList

----------------------------------------------------------------------------------------------------
@@TJvCustomPageListTreeView.PageLinks
Summary
  Used at design-time to set up the association between tree nodes and pages in a PageList.
Description
  This property is only used at design-time to set up the association between a node and the page that
  should be displayed y the IPageList implementation when that node is selected. To set up a page link
  association at run-time, type cast the tree node to TJvPageIndexNode and set its PageIndex property.

  \Example:
  <AUTOLINK OFF>
  <CODE>
    TJvPageIndexNode(JvCustomPageListTreeView1.Items[SomeIndex]).PageIndex := SomePageIndex; 
  </CODE>
  <AUTOLINK ON>

  The linking will only have effect if the PageList property has been assigned.
See Also
  TJvCustomPageListTreeView.PageList, TJvPageIndexNode.PageIndex

----------------------------------------------------------------------------------------------------
@@TJvCustomPageListTreeView.PageList
Summary
  Specifies the IPageList implementation.
Description
  Use the PageList property to associate an IPageList implementation with the treeview. If a PageList
  is assigned, the active page of the page list will change automatically when the user selects a new
  node in the treeview. The page that is activated is determined by the value of the selected nodes
  PageIndex property.

----------------------------------------------------------------------------------------------------
@@TJvCustomSettingsTreeView
Summary
  A treeview that behaves like the tree view in the Options Dialog in VS.net.
Description
  TJvCustomSettingsTreeView is a base class for treeviews that behave like the treeview in the Settings
  Dialog in Visual Studio: When a node in the treeview is selected, a new page of settings is shown on
  a panel to the right.

  Specifically, the following is true:

  * The normal ImageIndex/SelectedIndex is ignored for nodes - use PageNodeImages instead. You still
    need to assign a TImageList to the Images property
  * When a node is expanded, it is assigned the expanded image until it is co lapsed, regardless
    whether it's selected or not
  * When a parent folder is selected, the first non-folder child has its normal image set as the
    selected image
  * By default, AutoExpand and ReadOnly is true, ShowButtons and ShowLines are false

  Other than that, it should work like a normal TreeView. Note that the treeview was designed with
  AutoExpand = true in mind but should work with AutoExpand = false

  To get the VS look, Images should contain:

  * Image 0: Closed Folder
  * Image 1: Open Folder
  * Image 2: Right-pointing teal-colored arrow

  PageNodeImages should then be set to (the defaults):

  * ClosedFolder = 0;
  * ImageIndex = -1; (no image)
  * \OpenFolder = 1;
  * SelectedIndex = 2;

----------------------------------------------------------------------------------------------------
@@TJvCustomSettingsTreeView.OnGetImageIndex
Summary
  Occurs when the tree view looks up the ImageIndex of a node.
Description
  Write an OnGetImageIndex event handler to change the image index for the particular node before it is
  drawn. For example, the bitmap of a node can be changed to indicate a different state for the node.
See Also
  ImageIndex, Images

----------------------------------------------------------------------------------------------------
@@TJvCustomSettingsTreeView.OnGetSelectedIndex
Summary
  Occurs when the tree view looks up the SelectedIndex of a node.
Description
  Write an OnGetSelectedIndex event handler to change the selected image index of a node before it is
  drawn.
See Also
  Images, ImageIndex

----------------------------------------------------------------------------------------------------
@@TJvCustomSettingsTreeView.PageNodeImages
Summary
  Specifies the images to use for the different states of the nodes.
Description
  Use PageNodeImages to specify what images to use for the different states a node can be in. The
  recognized states are normal, selected, expanded and collapsed.
See Also
  Images

----------------------------------------------------------------------------------------------------
@@TJvPageIndexNode
Summary
  A tree node class that has an additional PageIndex property.
Description
  TJvPageIndexNode is the class for nodes created by the TJvPageListTreeView component.

----------------------------------------------------------------------------------------------------
@@TJvPageIndexNode.Assign
Summary
  Copies the properties of another node.
Description
  It the Source parameter is a TJvPageIndexNode object, Assign copies its properties to this node.
  Otherwise, Assign calls the inherited method so that any object that copies properties to a
  TJvPageIndexNode in its AssignTo method can do so.
Parameters
  Source - The object to copy.
See Also
  AssignTo

----------------------------------------------------------------------------------------------------
@@TJvPageIndexNode.PageIndex
Summary
  Specifies the page index of the node.
Description
  Use PageIndex to specify which page in an associated IPageList implementation should be activated
  when the node is selected.
See Also
  TJvCustomPage.PageList

----------------------------------------------------------------------------------------------------
@@TJvPageIndexNodes
Summary
  Maintains a list of TJvPageIndexNode nodes in a tree view control.
Description
  The Items property of the TJvCustomPageListTreeView is a TJvPageIndexNodes object and maintains the
  collection of nodes in the tree view. Nodes can be added, deleted, inserted, and moved within the
  tree view. Access the nodes in the tree view through the Items property of the tree view.

----------------------------------------------------------------------------------------------------
@@TJvPageLinks
Summary
  Class used to facilitate linking nodes to pages in a page list.
Description
  This class is only used at design-time and is provided to activate the design-time editor for setting
  up links between tree nodes and the pages in an IPageList implementation. Do not try to use this
  class or its methods as it isn't a "real" class.

----------------------------------------------------------------------------------------------------
@@TJvPageLinks.TreeView
Summary
  Specifies the TreeView that is the "parent" of the TJvPageLinks instance.
Description
  Write here a description

----------------------------------------------------------------------------------------------------
@@TJvPageListTreeView
<TITLEIMG TJvPageListTreeView>
JVCLInfo
  GROUP=JVCL.ListsAndTrees.Trees,JVCL.PagesAndTabs
  FLAG=Component
Summary
  A treeview that can be associated with an IPageList implementation.
Description
  TJvPageListTreeView is derived from TJvCustomPageListTreeView and publishes most of its properties
  but doesn't add any new functionality.

----------------------------------------------------------------------------------------------------
@@TJvSettingsTreeImages
Summary
  Property class for setting up usage of images from an image list.
Description
  TJvSettingsTreeImages is a property class that describes the images used in a
  TJvCustomSettingsTreeView as the usage of images in this control differs from a normal TreeView

----------------------------------------------------------------------------------------------------
@@TJvSettingsTreeImages.CollapsedIndex
Summary
  Specifies which image is displayed when a node with children is in its collapsed state.
Description
  Use the CollapsedIndex property with the Images property of the tree view to specify the image for
  the node in its collapsed state. If the node doesn't have any children, use the ImageIndex and
  SelectedIndex properties instead.
See Also
  TJvSettingsTreeImages.ExpandedIndex, TJvSettingsTreeImages.ImageIndex,
  TJvSettingsTreeImages.SelectedIndex

----------------------------------------------------------------------------------------------------
@@TJvSettingsTreeImages.ExpandedIndex
Summary
  Specifies which image is displayed when a node with children is in its expanded state.
Description
  Use the ExpandedIndex property with the Images property of the tree view to specify the image for the
  node in its expanded state. If the node doesn't have any children, use the ImageIndex and
  SelectedIndex properties instead.
See Also
  TJvSettingsTreeImages.CollapsedIndex, TJvSettingsTreeImages.ImageIndex,
  TJvSettingsTreeImages.SelectedIndex

----------------------------------------------------------------------------------------------------
@@TJvSettingsTreeImages.ImageIndex
Summary
  Specifies which image is displayed when a node is in its normal state and is not currently selected.
Description
  Use the ImageIndex property with the Images property of the tree view to specify the image for the
  node in its normal state. If the node has children, use the ExpandedIndex and CollapsedIndex
  properties instead.
See Also
  TJvSettingsTreeImages.CollapsedIndex, TJvSettingsTreeImages.ExpandedIndex,
  TJvSettingsTreeImages.SelectedIndex

----------------------------------------------------------------------------------------------------
@@TJvSettingsTreeImages.SelectedIndex
Summary
  Specifies which image is displayed when a node is in its selected state.
Description
  Use the SelectedIndex property with the Images property of the tree view to specify the image for the
  node in its selected state. If the node has children, use the ExpandedIndex and CollapsedIndex
  properties instead.
See Also
  TJvSettingsTreeImages.CollapsedIndex, TJvSettingsTreeImages.ExpandedIndex,
  TJvSettingsTreeImages.ImageIndex

----------------------------------------------------------------------------------------------------
@@TJvSettingsTreeImages.TreeView
Summary
  Specifies the treeview that this class is acting upon.
Description
  The TreeView property is used internally to reference the treeview the class is operating on. Only
  component developers might need to access this property.

----------------------------------------------------------------------------------------------------
@@TJvSettingsTreeView
<TITLEIMG TJvSettingsTreeView>
JVCLInfo
  GROUP=JVCL.ListsAndTrees.Trees
  FLAG=Component
Summary
  A treeview that behaves like the treeview in the Options Dialog in VS.Net.
Description
  TJvSettingsTreeView is derived from TJvCustomSettingsTreeView and publishes its properties but it
  does not add any new functionality.

