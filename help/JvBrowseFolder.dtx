----------------------------------------------------------------------------------------------------
@@JvBrowseFolder.pas
<GROUP JVCL.FileRef>
Summary
  Contains the TJvBrowseFolder component.
<INCLUDE JVCL.UnitText.dtx>
Author
  S�bastien Buysse
----------------------------------------------------------------------------------------------------
@@BrowseForComputer
JVCLInfo
  GROUP=JVCL.Dialogs
  FLAG=Routine,NotInParentList
Summary
  Brings up a dialog to allow the user to enter a computer name.
Description
  Call BrowseForComputer to let the user enter a computer name. 
Parameters
  ATitle       - Specifies a caption for the dialog.
  AllowCreate  - Specifies whether the user is allowed to create directories.
  ADirectory   - Returns the selected computer.
  AHelpContext - Specifies the help context ID number.
See Also
  BrowseForComputer, BrowseForFolder, TJvBrowseForFolderDialog
----------------------------------------------------------------------------------------------------
@@BrowseForFolder
JVCLInfo
  GROUP=JVCL.Dialogs
  FLAG=Routine,NotInParentList
Summary
  Brings up a dialog to allow the user to enter a directory name.
Description
  Call BrowseForFolder to let the user enter a directory name. 
Parameters
  ATitle       - Specifies a caption for the dialog.
  AllowCreate  - Specifies whether the user is allowed to create directories.
  ADirectory   - Returns the selected directory.
  AHelpContext - Specifies the help context ID number.
See Also
  TJvBrowseForFolderDialog
----------------------------------------------------------------------------------------------------
@@TFromDirectory
<TITLE TFromDirectory type>
Summary
  TFromDirectory defines values that describe specific folders.
Description
  The TFromDirectory type defines values describing specific folders, frequently used by applications, 
  but which may not have the same name or location on any given system.
  
  Use a TFromDirectory value to specify a specific folder in a unique system-independent way.
Note
  Some flags are not supported on all Shell32.dll versions, these flags are marked with a version number. 
  This version number indicates that the programming element was first implemented in that version and
  will also be found in all subsequent versions of the DLL. If no version number is specified, the 
  programming element is implemented in all versions. 
@@TFromDirectory.fdNoSpecialFolder
  ??
@@TFromDirectory.fdRootFolder
  The namespace root (the desktop folder).
@@TFromDirectory.fdRecycleBin
  The virtual folder containing the objects in the user's Recycle Bin.
@@TFromDirectory.fdControlPanel
  The virtual folder containing icons for the Control Panel applications.
@@TFromDirectory.fdDesktop
  The virtual folder representing the windows desktop, the root of the namespace.
@@TFromDirectory.fdDesktopDirectory
  The file system directory used to physically store file objects on the desktop (not to be confused
  with the desktop folder itself). A typical path is C:\\Documents and Settings\\<i>username</i>\\Desktop.
@@TFromDirectory.fdMyComputer
  The virtual folder representing My Computer, containing everything on the local computer: storage
  devices, printers, and Control Panel. The folder may also contain mapped network drives.
@@TFromDirectory.fdFonts
  A virtual folder containing fonts. A typical path is C:\\Windows\\Fonts.
@@TFromDirectory.fdNetHood
  A file system directory containing the link objects that may exist in the My Network Places virtual
  folder. It is not the same as fdNetwork, which represents the network namespace root. A typical
  path is C:\\Documents and Settings\\<i>username</i>\\NetHood.
@@TFromDirectory.fdNetWork
  A virtual folder representing Network Neighborhood, the root of the network namespace hierarchy.
@@TFromDirectory.fdPersonal
  The file system directory used to physically store a user's common repository of documents. A typical
  path is C:\\Documents and Settings\\<i>username</i>\\My Documents. This should be distinguished from the virtual
  My Documents folder in the namespace, identified by fdMyDocuments.
@@TFromDirectory.fdPrinters
  The virtual folder containing installed printers.
@@TFromDirectory.fdPrograms
  The file system directory that contains the user's program groups (which are themselves file system
  directories). A typical path is C:\\Documents and Settings\\<i>username</i>\\Start Menu\\Programs. 
@@TFromDirectory.fdRecent
  The file system directory that contains shortcuts to the user's most recently used documents. A typical
  path is C:\\Documents and Settings\\<i>username</i>\\My Recent Documents.
@@TFromDirectory.fdSendTo
  The file system directory that contains Send To menu items. A typical path is 
  C:\\Documents and Settings\\<i>username</i>\\SendTo.
@@TFromDirectory.fdStartMenu
  The file system directory containing Start menu items. A typical path is 
  C:\\Documents and Settings\\<i>username</i>\\Start Menu.
@@TFromDirectory.fdStartup
  The file system directory that corresponds to the user's Startup program group. The system starts these
  programs whenever any user logs onto windows NT or starts windows 95. A typical path is 
  C:\\Documents and Settings\\<i>username</i>\\Start Menu\\Programs\\Startup.
@@TFromDirectory.fdTemplates
  The file system directory that serves as a common repository for document templates. A typical path is
  C:\\Documents and Settings\\<i>username</i>\\Templates.
@@TFromDirectory.fdStartUpNonLocalized
  The file system directory that corresponds to the user's nonlocalized Startup program group.
@@TFromDirectory.fdCommonStartUpNonLocalized
  The file system directory that corresponds to the nonlocalized Startup program group for all users. 
  Valid only for windows NT systems.
@@TFromDirectory.fdCommonDocuments
  The file system directory that contains documents that are common to all users. A typical paths is
  C:\\Documents and Settings\\All Users\\Documents. Valid for windows NT systems, and windows 95 and 98 
  systems with Shfolder.dll installed.
@@TFromDirectory.fdCommonFavorites
  The file system directory that serves as a common repository for favorite items common to all
  users. Valid only for windows NT systems.
@@TFromDirectory.fdCommonPrograms
  The file system directory that contains the directories for the common program groups that appear
  on the Start menu for all users. A typical path is C:\\Documents and Settings\\All Users\\Start Menu\\Programs.
  Valid only for windows NT systems.
@@TFromDirectory.fdCommonStartUp
  The file system directory that contains the programs that appear in the Startup folder for all users.
  A typical path is C:\\Documents and Settings\\All Users\\Start Menu\\Programs\\Startup. Valid only
  for windows NT systems.
@@TFromDirectory.fdCommonTemplates
  The file system directory that contains the templates that are available to all users. A
  typical path is C:\\Documents and Settings\\All Users\\Templates. Valid only for windows NT systems.
@@TFromDirectory.fdCookies
  The file system directory that serves as a common repository for Internet cookies. A
  typical path is C:\\Documents and Settings\\<i>username</i>\\Cookies.
@@TFromDirectory.fdFavorites
  The file system directory that serves as a common repository for the user's favorite items.
  A typical path is C:\\Documents and Settings\\<i>username</i>\\Favorites.
@@TFromDirectory.fdHistory
  The file system directory that serves as a common repository for Internet history items.
@@TFromDirectory.fdInternet
  A virtual folder representing the Internet.
@@TFromDirectory.fdMyMusic
  The file system directory that serves as a common repository for music files. A typical
  path is C:\\Documents and Settings\\User\\My Documents\\My Music.
@@TFromDirectory.fdPrinthood
  The file system directory that contains the link objects that can exist in the \Printers
  virtual folder. A typical path is C:\\Documents and Settings\\<i>username</i>\\PrintHood.
@@TFromDirectory.fdConnections
  A virtual folder that contains network and dial-up connections. 
@@TFromDirectory.fdAppData
  <b>Version 4.71.</b> The file system directory that serves as a common repository for application-specific
  data. A typical path is C:\\Documents and Settings\\<i>username</i>\\Application Data.
@@TFromDirectory.fdInternetCache
  <b>Version 4.72.</b> The file system directory that serves as a common repository for temporary
  Internet files. A typical path is 
  C:\\Documents and Settings\\<i>username</i>\\Local Settings\\Temporary Internet Files.
@@TFromDirectory.fdAdminTools
  <b>Version 5.0.</b> The file system directory that is used to store administrative tools for an
  individual user.
@@TFromDirectory.fdCommonAdminTools
  <b>Version 5.0.</b> The file system directory containing administrative tools for all users of the
  computer.
@@TFromDirectory.fdCommonAppData
  <b>Version 5.0.</b> The file system directory containing application data for all users. A typical
  path is C:\\Documents and Settings\\All Users\\Application Data.
@@TFromDirectory.fdLocalAppData
  <b>Version 5.0.</b> The file system directory that serves as a data repository for local (nonroaming)
  applications. A typical path is C:\\Documents and Settings\\<i>username</i>\\Local Settings\\Application Data.
@@TFromDirectory.fdMyPictures
  <b>Version 5.0.</b> The file system directory that serves as a common repository for image files. A
  typical path is C:\\Documents and Settings\\<i>username</i>\\My Documents\\My Pictures.
@@TFromDirectory.fdProfile
  <b>Version 5.0.</b> The user's profile folder. A typical path is C:\\Documents and Settings\\<i>username</i>. 
  Applications should not create files or folders at this level; they should put their data under
  the locations referred to by fdAppData or fdLocalAppData.
@@TFromDirectory.fdProgramFiles
  <b>Version 5.0.</b> The Program Files folder. A typical path is C:\\Program Files.
@@TFromDirectory.fdProgramFilesCommon
  <b>Version 5.0.</b> A folder for components that are shared across applications. A typical path is 
  C:\\Program Files\\Common. Valid only for windows NT, 2000 and XP systems. Not 
  valid for windows Me.
@@TFromDirectory.fdSystem
  <b>Version 5.0.</b> The windows System folder. A typical path is C:\\Windows\\System32.
@@TFromDirectory.fdWindows
  <b>Version 5.0.</b> The windows directory or SYSROOT. This corresponds to the %windir% or %SYSTEMROOT%
  environment variables. A typical path is C:\\Windows.
@@TFromDirectory.fdCDBurnArea
  <b>Version 6.0.</b> The file system directory acting as a staging area for files waiting to be written 
  to CD. A typical path is 
  C:\\Documents and Settings\\<i>username</i>\\Local Settings\\Application Data\\Microsoft\\CD Burning.
@@TFromDirectory.fdCommonMusic
  <b>Version 6.0.</b> The file system directory that serves as a repository for music files common to 
  all users. A typical path is C:\\Documents and Settings\\All Users\\Documents\\My Music.
@@TFromDirectory.fdCommonPictures
  <b>Version 6.0.</b> The file system directory that serves as a repository for image files common to 
  all users. A typical path is C:\\Documents and Settings\\All Users\\Documents\\My Pictures.
@@TFromDirectory.fdCommonVideo
  <b>Version 6.0.</b> The file system directory that serves as a repository for video files common to 
  all users. A typical path is C:\\Documents and Settings\\All Users\\Documents\\My Videos.
@@TFromDirectory.fdMyDocuments
  <b>Version 6.0.</b> The virtual folder representing the My Documents desktop item. This should not 
  be confused with fdPersonal, which represents the file system folder that physically stores the documents.
@@TFromDirectory.fdMyVideo
  <b>Version 6.0.</b> The file system directory that serves as a common repository for video files. 
  A typical path is C:\\Documents and Settings\\<i>username</i>\\My Documents\\My Videos.
@@TFromDirectory.fdProfiles
  <b>Version 6.0.</b> The file system directory containing user profile folders. A typical path is
  C:\\Documents and Settings.
@@TFromDirectory.fdResources
  The windows system resource directory. A typical path is C:\\Windows\\Resources.
@@TFromDirectory.fdResourcesLocalized
  The windows localized system resource directory. A typical path is C:\\Windows\\Resources\<i>langid</id>.
@@TFromDirectory.fdCommonOEMLinks
  The folder containing links to OEM specific applications for all users.
@@TFromDirectory.fdComputersNearMe
  A virtual folder that contains links to nearby computers on the network. Nearness it is established 
  by common work group membership. 
----------------------------------------------------------------------------------------------------
@@TJvBrowsableObjectClass
<TITLE TJvBrowsableObjectClass type>
Summary
  TJvBrowsableObjectClass defines values describing the type of items in an enumeration
Description
  The TJvBrowsableObjectClass type defines values describing the type of items in an enumeration.
@@TJvBrowsableObjectClass.ocFolders
  \Include items that are folders in the enumeration. 
@@TJvBrowsableObjectClass.ocNonFolders
  \Include items that are not folders in the enumeration. 
@@TJvBrowsableObjectClass.ocIncludeHidden
  \Include hidden items in the enumeration. 
@@TJvBrowsableObjectClass.ocInitOnFirstNext
  allow EnumObject() to return before validating enum
@@TJvBrowsableObjectClass.ocNetPrinterSrch
  The caller is looking for printer objects. 
@@TJvBrowsableObjectClass.ocSharable
  The caller is looking for remote shares.
@@TJvBrowsableObjectClass.ocStorage
  \Include items with accessible storage and their ancestors.
----------------------------------------------------------------------------------------------------
@@TJvBrowsableObjectClasses
<TITLE TJvBrowsableObjectClasses type>
<COMBINE TJvBrowsableObjectClass>
----------------------------------------------------------------------------------------------------
@@TJvBrowseAcceptChange
<TITLE TJvBrowseAcceptChange type>
----------------------------------------------------------------------------------------------------
@@TJvBrowseForFolderDialog
<TITLEIMG TJvBrowseForFolderDialog>
JVCLInfo
  GROUP=JVCL.Dialogs
  FLAG=Component
Summary
  Displays a dialog box enabling the user to select a Shell folder.
Description
  It supports
  
  * Setting the root folder in the dialog box.
  * Setting you own instructions text.
  * Changing the OK button text.
  * Events that notify about selection changes and validation.
  * Customize filtering (Only for windows XP)
----------------------------------------------------------------------------------------------------
@@TJvBrowseForFolderDialog.Directory
Summary
  Indicates the directory path of the last folder selected.
Description
  The Directory property returns the path of the most recently selected item. 

  To make an item selected by default in the dialog�s box, assign a value to Directory in the Object Inspector 
  or in program code. Programmatic changes to Directory have no effect while the dialog is active, use
  <LINK SetSelection@string,SetSelection> for this purpose.
See Also
  DisplayName, SetSelection
----------------------------------------------------------------------------------------------------
@@TJvBrowseForFolderDialog.DisplayName
Summary
  Indicates the display name of the folder selected by the user. 
Description
  Read DisplayName to retrieve the display name of the folder selected by the user after a call to
  Execute. Alternatively read Directory to retrieve the path of the folder the user selected.
See Also
  Directory
----------------------------------------------------------------------------------------------------
@@TJvBrowseForFolderDialog.Execute
Summary
Description
See Also
----------------------------------------------------------------------------------------------------
@@TJvBrowseForFolderDialog.Handle
Summary
Description
See Also
----------------------------------------------------------------------------------------------------
@@TJvBrowseForFolderDialog.HelpContext
Summary
  Specifies the context number for online Help.
Description
  The HelpContext property is an integer value that determines which Help screen appears when the 
  user requests context-sensitive online Help. 
----------------------------------------------------------------------------------------------------
@@TJvBrowseForFolderDialog.LastPidl
Summary
Description
See Also
----------------------------------------------------------------------------------------------------
@@TJvBrowseForFolderDialog.OnAcceptChange
<COMBINEWITH TJvBrowseAcceptChange>
Summary
  Occurs after the selection has changed in the Browse dialog box.
Description
  Write an OnAcceptChange event handler to enable or disable the Browse dialog's OK button, depending
  on the newly selected item.
Parameters
  Sender    - The TJvBrowseForFolderDialog component that has sent this event.
  NewFolder - Specifies the newly selected item. 
  Accept    - Set to True to enable the dialog's OK button, set to False to disable the dialog's OK
              button.
See Also
  OnChange, OnValidateFailed, SetOKEnabled
----------------------------------------------------------------------------------------------------
@@TJvBrowseForFolderDialog.OnChange
<COMBINEWITH TJvDirChange>
Summary
  Occurs after the selection has changed in the Browse dialog box.
Description
  Write an OnChange event handler to take specific actions when the user selects an item in the Browse
  dialog box. For example you can call <LINK TJvBrowseForFolderDialog.SetStatusText, SetStatusText> to
  change the status text depending on the newly selected item.
  
  Alternatively use the OnAcceptChange event to enable or disable the Browse dialog's OK button, depending
  on the newly selected item.
Parameters
  Sender    - The TJvBrowseForFolderDialog component that has sent this event.
  Directory - Specifies the newly selected item. 
See Also
  OnAcceptChange, OnValidateFailed, SetOKEnabled
----------------------------------------------------------------------------------------------------
@@TJvBrowseForFolderDialog.OnGetEnumFlags
<COMBINEWITH TJvGetEnumFlagsEvent>
Summary
  Occurs for each Shell folder before the dialog displays it.
Description
  Write an OnGetEnumFlags event handler to specify which classes of objects in a Shell folder should
  be enumerated.
Note
  Only for windows XP systems.
Parameters
  Sender  - The TJvBrowseForFolderDialog component that has sent this event.
  AFolder - Indicates the Shell folder that is to be displayed.
  Flags   - Set Flags to the classes of objects that should be enumerated in the folder specified by
            AFolder.
See Also
  OnShouldShow
----------------------------------------------------------------------------------------------------
@@TJvBrowseForFolderDialog.OnInitialized
Summary
  Occurs after the Browse dialog box has finished initializing.
Description
  Write an OnInitialized event handler to respond when the Browse dialog has finished initializing.
----------------------------------------------------------------------------------------------------
@@TJvBrowseForFolderDialog.OnShouldShow
<COMBINEWITH TJvShouldShowEvent>
Summary
  Occurs for each item in a folder before the dialog displays it.
Description
  Write an OnShouldShow event handler to specify which individual items of a folder should be displayed 
  in the dialog.
Note
  Only for windows XP systems.
Parameters
  Sender - The TJvBrowseForFolderDialog component that has sent this event.
  Item   - Indicates the item that is to be displayed.
  DoShow - Set to True to have the item displayed, set to False to prevent the item from being displayed.
See Also
  OnGetEnumFlags
----------------------------------------------------------------------------------------------------
@@TJvBrowseForFolderDialog.OnValidateFailed
<COMBINEWITH TJvValidateFailedEvent>
Summary
  Occurs when the user typed an invalid name into the edit box of the Browse dialog box.
Description
  Write an OnValidateFailed event handler to respond when the user types an invalid name
  into the edit box of the Browse dialog box. 
  
  Set the CanCloseDialog parameter to False to prevent the dialog from closing. The OnValidateFailed 
  event handler is responsible for telling the user why the dialog doesn�t close.
  
  Flags odEditBox and odValidate must be included in Options.
Note
  Only for shell versions 4.71 and up.
Parameters
  Sender         - The TJvBrowseForFolderDialog component that has sent this event.
  AEditText      - Specifies the invalid name.
  CanCloseDialog - Set to True to allow the dialog to be dismissed, set to False to keep the dialog
                   displayed.
See Also
  OnAcceptChange, OnChange, OnValidateFailed, SetOKEnabled
----------------------------------------------------------------------------------------------------
@@TJvBrowseForFolderDialog.Options
Summary
  Determines the appearance and behavior of the Browse dialog box.
Description
  Use the Options property to customize the appearance and functionality of the dialog. Use these flags to
  specify:
  
  * Which items should be displayed in the dialog. (Folders, \Printers, URL's)
  * Which visual controls should be displayed in the dialog. (Edit control, "\New Folder" button,
    usage hint label)
  * Whether the new user interface should be used.
  * Whether names entered into the edit box should be validated.
  
  Note that the component enforces a correct setting of Options, thus by including a flag in Options,
  some more flags may be included or excluded from Options.
See Also
  Directory
----------------------------------------------------------------------------------------------------
@@TJvBrowseForFolderDialog.Position
Summary
  Represents the placement of the dialog box.
Description
  Use Position to get or set the placement of the Browse dialog box.
----------------------------------------------------------------------------------------------------
@@TJvBrowseForFolderDialog.RootDirectory
Summary
  Specifies the location of the root folder from which to start browsing. 
Description
  Use RootDirectory to specify a special folder as root folder from which to start browsing. Only 
  the specified folder and any subfolders that are beneath it in the namespace hierarchy will appear 
  in the dialog box.
  
  Property RootDirectory can be used to specify a specific folder which may not have the same name or
  location on any given system. For example, the system folder may be "C:\Windows" on one system and
  "C:\Winnt" on another.

  Setting RootDirectory to fdNoSpecialFolder specifies that RootDirectoryPath should be used as root
  folder for the dialog.
See Also
  Directory, RootDirectoryPath
----------------------------------------------------------------------------------------------------
@@TJvBrowseForFolderDialog.RootDirectoryPath
Summary
  Specifies the location of the root folder from which to start browsing. 
Description
  Use RootDirectoryPath to specify the location of the root folder from which to start browsing. Only 
  the specified folder and any subfolders that are beneath it in the namespace hierarchy will appear 
  in the dialog box.
  
  Use RootDirectory instead of RootDirectoryPath to specify the root directory in a unique 
  system-independent way.
  
  RootDirectory must be set to fdNoSpecialFolder, otherwise the value of RootDirectoryPath is ignored.
See Also
  Directory, RootDirectory
----------------------------------------------------------------------------------------------------
@@TJvBrowseForFolderDialog.SetExpanded@PItemIDList
<COMBINE TJvBrowseForFolderDialog.SetExpanded@string>
----------------------------------------------------------------------------------------------------
@@TJvBrowseForFolderDialog.SetExpanded@string
Summary
  Expands a path in the Browse dialog box.
Description
  Call SetExpanded or SetExpandedW to expand a path in the Browse dialog. The path to expand can be
  specified as pointer to an item identifier list (PIDL), as a \string or as a wide \string.

  Calls to SetExpanded or SetExpandedW do have no effect while the dialog is <I>not</I> active.
Parameters
  APath  - Specifies the path to expand as a \string or as a wide \string.
  IDList - Specifies the path to expand as pointer to an item identifier list.
See Also
  SetSelection
----------------------------------------------------------------------------------------------------
@@TJvBrowseForFolderDialog.SetExpandedW
<COMBINE TJvBrowseForFolderDialog.SetExpanded@string>
----------------------------------------------------------------------------------------------------
@@TJvBrowseForFolderDialog.SetOKEnabled
Summary
  Enables or disables the Browse dialog box's OK button. 
Description
  Call SetOKEnabled to enable or disable the Browse dialog box's OK button. For example
  you can call SetOKEnabled in an OnChange event handler to enable or disable the button depending on 
  the newly selected item; but for this specific purpose you can also use OnAcceptChange.

  Calls to SetOKEnabled do have no effect while the dialog is <i>not</i> active.
Parameters
  Value - Set to True to enable the OK button. To disable the OK button, set Value to False.
See Also
  OnAcceptChange, OnChange, OnValidateFailed, SetOKText
----------------------------------------------------------------------------------------------------
@@TJvBrowseForFolderDialog.SetOKText
Summary
  Sets the text that is displayed on the Browse dialog box's OK Button.
Description
  Call SetOKText to sets the text that is displayed on the Browse dialog box's OK button. For example
  you can call SetOKText in an OnInitialized event handler, or call it in an OnChange to set the
  button text a specific text depending on the newly selected item.

  Calls to SetOKText do have no effect while the dialog is <i>not</i> active.
Parameters
  AText - Specifies the desired text for the OK Button.
----------------------------------------------------------------------------------------------------
@@TJvBrowseForFolderDialog.SetOKTextW
<COMBINE TJvBrowseForFolderDialog.SetOKText>
----------------------------------------------------------------------------------------------------
@@TJvBrowseForFolderDialog.SetSelection@PItemIDList
<COMBINE TJvBrowseForFolderDialog.SetSelection@string>
----------------------------------------------------------------------------------------------------
@@TJvBrowseForFolderDialog.SetSelection@string
Summary
  Selects a folder in the Browse dialog box.
Description
  Call SetSelection to select a folder in the Browse dialog. The folder to select can be specified as
  pointer to an item identifier list (PIDL) or as a \string. For example you can call SetSelection in an
  OnInitialized event handler, but for this specific purpose you can also use property Directory.

  Calls to SetSelection do have no effect while the dialog is <I>not</I> active, use property Directory
  for this purpose.
Parameters
  APath  - Specifies the path to expand as a \string.
  IDList - Specifies the path to expand as pointer to an item identifier list.
See Also
  SetExpanded
----------------------------------------------------------------------------------------------------
@@TJvBrowseForFolderDialog.SetStatusText
Summary
  Sets the status text in the Browse dialog box.
Description
  Call SetStatusText to change the \string that is displayed above the tree view control in the dialog box. 

  Calls to SetStatusText or SetStatusTextW do have no effect while the dialog is <i>not</i> active,
  use property StatusText for this purpose.
  
  Flag odNewDialogStyle must <i>not</i> be included in Options, otherwise this procedure will fail.
Parameters
  AText - Specifies the desired status text.
See Also
  StatusText, Title
----------------------------------------------------------------------------------------------------
@@TJvBrowseForFolderDialog.SetStatusTextW
<COMBINE TJvBrowseForFolderDialog.SetStatusText>
----------------------------------------------------------------------------------------------------
@@TJvBrowseForFolderDialog.StatusText
Summary
  Specifies the \string that is displayed above the tree view control in the dialog box.
Description
  Use StatusText in combination with Title to specify a \string that is displayed above the tree view
  control in the dialog box. This \string can be used to specify instructions to the user.

  Title may consist of two lines of text. You can add another line by including flag odStatusAvailable
  in Options, and setting property StatusText.

  Setting StatusText to an empty \string and including flag odStatusAvailable in Options, will display
  the path of the last selected directory above the tree view control. (Only if odNewDialogStyle is <I>not</I>
  included in Options)

  You can also call <LINK TJvBrowseForFolderDialog.SetStatusText, SetStatusText> when the dialog is
  active to change the status text.

  Flag odStatusAvailable must be included in Options and odNewDialogStyle must be excluded from Options,
  otherwise the value of this property is ignored.
See Also
  Options, Title
----------------------------------------------------------------------------------------------------
@@TJvBrowseForFolderDialog.Title
Summary
  Specifies the \string that is displayed above the tree view control in the dialog box. 
Description
  Use Title in combination with StatusText to specify a \string that is displayed above the tree view
  control in the dialog box. This \string can be used to specify instructions to the user.
  
  Title may consist of two lines of text. You can add another line by including flag odStatusAvailable
  in Options, and setting property StatusText.
See Also
  StatusText
----------------------------------------------------------------------------------------------------
@@TJvDirChange
<TITLE TJvDirChange type>
----------------------------------------------------------------------------------------------------
@@TJvFolderPos
<TITLE TJvFolderPos type>
Summary
  TJvFolderPos defines values describing the position of a dialog. 
Description
  The TJvFolderPos type defines values describing the position of a dialog.
@@TJvFolderPos.fpDefault
  The form appears in a position on the screen and with a height and width determined by the operating system. 
@@TJvFolderPos.fpScreenCenter
  The dialog is positioned in the center of the screen. 
@@TJvFolderPos.fpFormCenter
  The dialog is positioned in the center of the form that is the owner of the component. If the component
  has no form as owner it's positioned in the center of the application�s main form. 
@@TJvFolderPos.fpTopLeft
  The dialog is positioned in the upper left corner of the screen. 
@@TJvFolderPos.fpTopRight
  The dialog is positioned in the upper right corner of the screen. 
@@TJvFolderPos.fpBottomLeft
  The dialog is positioned in the lower left corner of the screen. 
@@TJvFolderPos.fpBottomRight
  The dialog is positioned in the lower right corner of the screen. 
----------------------------------------------------------------------------------------------------
@@TJvGetEnumFlagsEvent
<TITLE TJvGetEnumFlagsEvent type>
----------------------------------------------------------------------------------------------------
@@TJvShouldShowEvent
<TITLE TJvShouldShowEvent type>
----------------------------------------------------------------------------------------------------
@@TJvValidateFailedEvent
<TITLE TJvValidateFailedEvent type>
----------------------------------------------------------------------------------------------------
@@TOptionsDir
<TITLE TOptionsDir type>
<COMBINE TOptionsDirectory>
----------------------------------------------------------------------------------------------------
@@TOptionsDirectory
<TITLE TOptionsDirectory type>
Summary
  TOptionsDir and TOptionsDirectory determine the behavior of a folder selection dialog. 
Description
  TOptionsDirectory values determine the appearance and behavior of a folder selection dialog. 

  \TOptionsDir is a set of TOptionsDirectory values.
Note
  Some flags are not supported on all Shell32.dll versions, these flags are marked with a version number. 
  This version number indicates that the programming element was first implemented in that version and
  will also be found in all subsequent versions of the DLL. If no version number is specified, the 
  programming element is implemented in all versions. 
@@TOptionsDirectory.odBrowseForComputer
  Only returns computers. If the user selects anything other than a computer, the OK button is grayed.
@@TOptionsDirectory.odOnlyDirectory
  Not used.
@@TOptionsDirectory.odOnlyPrinters
  Only returns printers. If the user selects anything other than a printer, the OK button is grayed.
@@TOptionsDirectory.odNoBelowDomain
  Does not include network folders below the domain level in the dialog box's tree view control.
@@TOptionsDirectory.odSystemAncestorsOnly
  Only returns file system ancestors. An ancestor is a subfolder that is beneath the root folder in the
  namespace hierarchy. If the user selects an ancestor of the root folder that is not part of the file
  system, the OK button is grayed.
@@TOptionsDirectory.odFileSystemDirectoryOnly
  Only returns file system directories. If the user selects folders that are not part of the file system,
  the OK button is grayed.
@@TOptionsDirectory.odStatusAvailable
  Includes a status area in the dialog box. When property StatusText is set to an empty \string then the
  component will display the path of the last selected directory above the tree view control. Otherwise
  the text specified by StatusText will be displayed. Flag odNewDialogStyle must <i>not</i> be set.
@@TOptionsDirectory.odIncludeFiles
  <b>Version 4.71.</b> The Browse dialog box will display files as well as folders.
@@TOptionsDirectory.odIncludeUrls
  <b>Version 5.0.</b> The Browse dialog box can display URLs. 
@@TOptionsDirectory.odEditBox
  <b>Version 4.71.</b> Includes an edit control in the Browse dialog box that allows the user to type the 
  name of an item.
@@TOptionsDirectory.odNewDialogStyle
  <b>Version 5.0.</b> Uses the new user interface. Setting this flag provides the user with a larger dialog box that 
  can be resized. The dialog box has several new capabilities including: drag and drop capability within 
  the dialog box, reordering, shortcut menus, new folders, delete, and other shortcut menu commands.
@@TOptionsDirectory.odShareable
  <b>Version 5.0.</b> The Browse dialog box can display shareable resources on remote systems. It is intended
  for applications that want to expose remote shares on a local system. The odNewDialogStyle flag must 
  also be set.
@@TOptionsDirectory.odUsageHint
  <b>Version 6.0.</b> When combined with odNewDialogStyle, adds a usage hint to the dialog box in place of
  the edit box. Flag odEditBox overrides this flag.
@@TOptionsDirectory.odNoNewButtonFolder
  <b>Version 6.0.</b> Does not include the "\New Folder" button in the Browse dialog box.
@@TOptionsDirectory.odValidate
  <b>Version 4.71.</b> If the user types an invalid name into the edit box, the Browse dialog box will trigger
  the OnValidateFailed event. Flag odEditBox must also be set.
