##Package: Cmp
##Status: Completed,Checked (peter3)
----------------------------------------------------------------------------------------------------
@@JvTranslator.pas
Summary
  Contains the TJvTranslator component.
<INCLUDE JVCL.UnitText.dtx>
Author
  S�bastien Buysse

----------------------------------------------------------------------------------------------------
@@TJvTranslator
<TITLEIMG TJvTranslator>
JVCLInfo
  GROUP=JVCL.FormsAndApp.Forms,JVCL.Internet.ConvertersAndParsers.XML
  FLAG=Component
Summary
  Translates an application, form or component.
Description
  TJvTranslator is a non-visual component that can be used to automatically change the properties of
  forms and components. The forms and components to change and what value to change their properties to
  can be read from a string, a stream or a file.

  You can also use this component to read the components and their properties and create an \XML string
  from it. You can then use the generated \XML to create a template used in translation.

----------------------------------------------------------------------------------------------------
@@TJvTranslator.ClearSkipList
Summary
  Clears the skip list.
Description
  Call ClearSkipList to clear the internal skip list.
See Also
  <LINK SkipClass>, <LINK SkipProperty>

----------------------------------------------------------------------------------------------------
@@TJvTranslator.ComponentToXML
Summary
  Converts the properties of a TComponent to an \XML string.
Description
  Use ComponentToXML to convert the name and published properties of a TComponent instance into a
  correctly formatted \XML string compatible with the format used by JvTranslator. You can create a
  complete translation template file for your application by calling ComponentToXML for the Application
  object.

  Note that ComponentToXML lists all published properties not skipped by a call to SkipProperty of a
  component and you might have to clean the returned \XML from properties you don't need to translate.
Parameters
  AComponent - The component to create the \XML fragment for. If AComponent is found in the skip list,
                its properties will not be listed. If any of the component's properties is found in the
                skip list, the property will not be listed. If AComponent is Application, a root node
                with the name 'Translation' is automatically added to the returned \XML.
  Recurse    - Set this parameter to true to recurse the Components array of AComponent and create the
                \XML for owned components as well. If AComponent is in the skip list and Recurse is
                true, the owned components of AComponent will still be checked and its properties added
                to the \XML string.
Return value
  An \XML string then can be used in calls to TJvTranslator.Translate or as a building block for a XML
  translation template file.
See Also
  <LINK SkipClass>, <LINK SkipProperty>, InSkipList

----------------------------------------------------------------------------------------------------
@@TJvTranslator.InSkipList@TClass
Summary
  Checks if a specific class is skipped when creating \XML with ComponentToXML
Description
  InSkipList returns true if the specified class has been added to the skip list by a call to
  SkipClass. If a class is in the skip list, its properties will not be included in the \XML string.
Parameters
  AClass - The class to check. The specified class and all classes inherited from it are checked.
Return value
  True if AClass is in the skip list, false if it is not.
See Also
  <LINK SkipClass>

----------------------------------------------------------------------------------------------------
@@TJvTranslator.InSkipList@TClass@string
Summary
  Checks if a specific class is skipped when creating \XML with ComponentToXML
Description
  InSkipList returns true if the specified class has been added to the skip list by a call to
  SkipClass. If a class is in the skip list, its properties will not be included in the \XML string.
Parameters
  AClass   - The class to check. The specified class and all classes inherited from it are checked.
  PropName - The name of the property to check.
Return value
  True if PropName of AClass is in the skip list, false if it is not.
See Also
  <LINK SkipClass>, <LINK SkipProperty>

----------------------------------------------------------------------------------------------------
@@TJvTranslator.InSkipList@TObject
Summary
  Checks if a specific class instance is skipped when creating \XML with ComponentToXML
Description
  InSkipList returns true if the specified class instance has been added to the skip list by a call to
  SkipClass. If a class is in the skip list, its properties will not be included in the \XML string.
Parameters
  Obj - The class instance to check.
Return value
  True if Obj is in the skip list, false if it is not.
See Also
  <LINK SkipClass>

----------------------------------------------------------------------------------------------------
@@TJvTranslator.InSkipList@TObject@string
Summary
  Checks if a specific class is skipped when creating \XML with ComponentToXML
Description
  InSkipList returns true if the specified class has been added to the skip list by a call to
  SkipClass. If a class is in the skip list, its properties will not be included in the \XML string.
Parameters
  Obj      - The class instance to check.
  PropName - The name of the property.
Return value
  True if PropName of Obj is in the skip list, false if it is not.
See Also
  <LINK SkipClass>

----------------------------------------------------------------------------------------------------
@@TJvTranslator.SkipClass
Summary
  Adds a class to the internal skip list.
Description
  Call SkipClass with a class name to add the class to the skip list. The skip list is used by
  ComponentToXML to determine whether the properties of a class should be added to the returned \XML
  string and by Translate to determine what to \translate.

  Note that even if a class is in the skip list, the Components array of the skipped class will still
  be iterated and checked if the Recurse parameter to ComponentToXML is true.
Parameters
  AClass - The class to skip. Calling SkipClass with the same class multiple times, does nothing.
See Also
  InSkipList

----------------------------------------------------------------------------------------------------
@@TJvTranslator.SkipProperty
Summary
  Adds a class property to the internal skip list.
Description
  Call SkipProperty with a class and property name to add the property of the class to the skip list.
  The skip list is used by ComponentToXML to determine whether the properties of a class should be
  added to the returned \XML string and by Translate to determine what to \translate.

  Note that even if a class is in the skip list, the Components array of the skipped class will still
  be iterated and checked if the Recurse parameter to ComponentToXML is true.
Parameters
  AClass   - The class containing the property.
  PropName - the property to skip in the class.
See Also
  InSkipList

----------------------------------------------------------------------------------------------------
@@TJvTranslator.Translate@string
Summary
  Translates the entire application.
Description
  Translates all forms in the current application using the \XML in FileName. Only instantiated forms
  and their components are translated.
Parameters
  FileName - The name of the file containing the translation.

----------------------------------------------------------------------------------------------------
@@TJvTranslator.Translate@string@string
Summary
  Translates a string using the currently loaded \XML.
Description
  Searches the currently loaded \XML for a node matching the names as defined by Category and Item and
  \returns the found value.

  <B>\Example</B>

  If Category = "Variables" and Item = "Item", the \XML would have to look like the following for this
  method to find a match:
  <AUTOLINK OFF>
  <CODE>
    \<Translation\>
      ...
      \<Variables\>
        \<Item\>XXXXX\</Item\>
      \</Variables\>
      ...
    \</Translation\>
  </CODE>
  <AUTOLINK ON>
  This would return the string "XXX". Alternatively, the \XML could look like this:
  <AUTOLINK OFF>
  <CODE>
    \<Translation\>
      ...
      \<Variables\>
        \<Item Value="YYYY" /\>
      \</Variables\>
      ...
    \</Translation\>
  </CODE>
  <AUTOLINK ON>
  This would return the string "YYY". if more than one node has the same name, the first one found is
  the one used.
Parameters
  Category - The name of the Category of the translation to use. The value of this parameter is
              checked against the names of the direct sub nodes of the root node in the \XML.
  Item     - The name of the Item of the translation to use. The value of this parameter is checked
              against the names of the subnodes.
Return value
  \Returns the value of the Item node or, if not found, the value of a property of Item named "Value"
  \or an empty string if the node was not found.

----------------------------------------------------------------------------------------------------
@@TJvTranslator.Translate@string@TCustomForm
Summary
  Translates a form.
Description
  Translates a form and all contained components on the form using the translation loaded from
  filename. If the file cannot be found, this function raises an exception.
Parameters
  FileName - The name of the file containing the translation.
  Form     - The form to translate.

----------------------------------------------------------------------------------------------------
@@TJvTranslator.Translate@TCustomForm
Summary
  Translates a form.
Description
  Translates a form and all contained components on the form using the currently loaded \XML. The XML
  can be loaded with any of the Translate functions that contains a string, a filename or a stream.
Parameters
  Form - The form to translate.

----------------------------------------------------------------------------------------------------
@@TJvTranslator.Translate@TStream
Summary
  Translates the entire application.
Description
  Translates all forms in the current application using the \XML in the stream. Only instantiated forms
  and their components are translated.
Parameters
  Stream - The stream to read the \XML from. The reading starts at the current position in the
            stream.

----------------------------------------------------------------------------------------------------
@@TJvTranslator.Translate@TStream@TCustomForm
Summary
  Translates a form.
Description
  Translates a form and all contained components on the form using the translation loaded from Stream.
Parameters
  Stream - The stream to read the \XML from. The reading starts at the current position in the
            stream.
  Form   - The form to translate.

----------------------------------------------------------------------------------------------------
@@TJvTranslator.TranslateScreen@string
Summary
  Translates all forms on the screen.
Description
  Translates all forms maintained by the global Screen object. Only instantiated forms and their
  components are translated.
Parameters
  FileName - The name of the file containing the translation.

----------------------------------------------------------------------------------------------------
@@TJvTranslator.TranslateScreen@TStream
Summary
  Translates all forms on the screen.
Description
  Translates all forms maintained by the global Screen object. Only instantiated forms and their
  components are translated.
Parameters
  Stream - The stream to read the \XML from. The reading starts at the current position in the
            stream.

----------------------------------------------------------------------------------------------------
@@TJvTranslator.TranslateScreenString
Summary
  Translates all forms on the screen.
Description
  Translates all forms maintained by the global Screen object. Only instantiated forms and their
  components are translated.
Parameters
  S - The string to read the \XML from.

----------------------------------------------------------------------------------------------------
@@TJvTranslator.TranslateString@string
Summary
  Translates all forms on the screen.
Description
  Translates all forms maintained by the global Screen object. Only instantiated forms and their
  components are translated.
Parameters
  S - The string to read the \XML from.

----------------------------------------------------------------------------------------------------
@@TJvTranslator.TranslateString@string@TCustomForm
Summary
  Translates a form.
Description
  Translates a form and all contained components on the form using the \XML in the string.
Parameters
  S    - The string to read the \XML from.
  Form - The form to translate.

----------------------------------------------------------------------------------------------------
@@TJvTranslator.UnskipClass
Summary
  Removes a class from the internal skip list.
Description
  Call UnskipClass to remove a class from the skip list.
Parameters
  AClass - The class to remove
See Also
  TJvTranslator.UnskipProperty

----------------------------------------------------------------------------------------------------
@@TJvTranslator.UnskipProperty
Summary
  Removes a property from the internal skip list.
Description
  Call UnskipProperty to remove a property from the internal skip list. Removing the same property
  several times does nothing.
Parameters
  AClass   - The class containing the property.
  PropName - The name of the property.
See Also
  TJvTranslator.UnskipClass

----------------------------------------------------------------------------------------------------
@@TJvTranslator.XML
Summary
  Provides access to the internal \XML object.
Description
  Use the \XML property to access the internal TJvSimpleXML object used to read and write the \XML
  data.

----------------------------------------------------------------------------------------------------
@@TJvTranslatorStrings
<TITLEIMG TJvTranslatorStrings>
JVCLInfo
  GROUP=JVCL.FormsAndApp.Forms,JVCL.Internet.ConvertersAndParsers.XML
  FLAG=Component
Summary
  Stores translatable strings that are not part of a form or component, like variables etc.
Description
  To be able to translate string variables not part of a class, you can use the TJvTranslatorStrings
  component. Drop it on a form (one per application is enough) and, at run-time, add the strings you
  want to translate by using the Add method. Note that the strings you add must be declared as global
  variables somewhere in your project.

  <B>\Example</B>
  <AUTOLINK OFF>
  <CODE>
  var AString:string = 'This is a string';

  ...
    // adding to list:
    JvTranslatorStrings.Add('AString',AString);

    // using:
    ShowMessage(AString); // this will show the translated value of AString
                          // if it was included in the translation data
                          // loaded into a TJvTranslator instance 
  </CODE>
  <AUTOLINK ON>

----------------------------------------------------------------------------------------------------
@@TJvTranslatorStrings.Add
Summary
  Adds a new Name/\Value pair to the translation list.
Description
  Use Add to add Name/\Value pair of strings that you want to be able to translate. The \Value must be
  available as a global variable somewhere in your project since TJvTranslatorStrings only stores a
  reference to it (that is how it can transalte it automatically). When you use the string, use the
  variable, so you get the translated value of the string.
Parameters
  Name  - The name of the string variable. This value must be unique within the application.
  Value - The value of the string.
Return value
  The index into the list where the Name/\Value pair was saved.

----------------------------------------------------------------------------------------------------
@@TJvTranslatorStrings.Count
Summary
  \Returns the number of items in the list.
Description
  Count contains the number of items in the Strings and Value array. Since Strings and Value are
  indexed starting with 0, the value of Count is always one greater than the index of the last member
  \of Strings and Value.

----------------------------------------------------------------------------------------------------
@@TJvTranslatorStrings.IndexOf
Summary
  \Returns the position of a string in the list.
Description
  Call IndexOf to obtain the position of the first occurrence of the string Name, or of a string that
  differs from Name only by case. IndexOf returns the 0-based index of the string. Thus, if Name
  matches the first string in the list, IndexOf returns 0, if Name is he second string, IndexOf
  \returns 1, and so on. If the string is not in the list, IndexOf returns -1.
Parameters
  Name - The string to search for.
Return value
  The 0-based index of the found string or -1 if not found.
See Also
  TJvTranslatorStrings.Strings, TJvTranslatorStrings.Value

----------------------------------------------------------------------------------------------------
@@TJvTranslatorStrings.Strings
Summary
  Lists the names, referenced by a 0-based index.
Description
  Use Strings to read or modify the string at a particular position. Index gives the position of the
  string, where 0 is the position of the first string, 1 is the position of the second string, and so
  \on. To locate a particular string in the list, call the <LINK IndexOf> method.

  \Note that this property does not work like you might expect. Specifically, when reading, the
  returned value is the <I>Name</I> part of the string, but when writing the value is written to the
  <I>\Value</I> part of the string. To read the value, use Value instead.
See Also
  <LINK Add>, <LINK IndexOf>

----------------------------------------------------------------------------------------------------
@@TJvTranslatorStrings.Value
Summary
  Lists the values, referenced by a 0-based index.
Description
  Use Value to read the value of a string at a particular position. Index gives the position of the
  string, where 0 is the position of the first string, 1 is the position of the second string, and so
  \on.

