##Package: Custom
##Status: Completed,Checked (peter3)
##Skip: TJvImageLoadEvent
----------------------------------------------------------------------------------------------------
@@JvImagesViewer.pas
Summary
  Contains the TJvImagesViewer component.
<INCLUDE JVCL.UnitText.dtx>
Author
  Peter Th�rnqvist

----------------------------------------------------------------------------------------------------
@@TJvImageItem
Summary
  Defines an item in the Items array of a TJvImagesViewer control.
Description
  The TJvImagesViewer displays images either from disk or from other sources. Each item in its Items
  array is of type TJvImageItem. Use the properties and methods of TJvImageItem to modify how the image
  is displayed in the TJvImagesViewer control.

----------------------------------------------------------------------------------------------------
@@TJvImageItem.Caption
Summary
  Specifies the text to display for the item.
Description
  Use Caption to specify the text to display along with the image.

----------------------------------------------------------------------------------------------------
@@TJvImageItem.FileName
Summary
  Specifies the file name of the file on disk the item is displaying.
Description
  Use Filename to set the file on disk the item should be displaying. When changing Filename, the item
  is not updated directly. Instead, Picture is set to nil and the next time the TJvImagesViewer control
  requests the image, it is loaded from disk. This means that an item that is never displayed, will not
  be loaded at all.
See Also
  TJvImageItem.Picture

----------------------------------------------------------------------------------------------------
@@TJvImageItem.Picture
Summary
  Provides access to the image the item is displaying.
Description
  Use the Picture property to access the image the item is displaying or to modify the properties of
  the image. For example, if you want to dis lay images that are not on disk, assign an image from
  another source to the Picture property at run-time.
See Also
  TJvImageItem.FileName

----------------------------------------------------------------------------------------------------
@@TJvImageItem.ReduceMemoryUsage
Summary
  Minimizes memory usage.
Description
  ReduceMemoryUsage minimizes the memory used by the item by freeing the Picture object but only if
  Filename specifies an existing file (no check is done to determine if Filename is a valid image
  file). When the Picture property of the item is accessed the next time, the image is recreated from
  disk.

  You don't have to call ReduceMemoryUsage directly. It is called automatically by the TJvImagesViewer
  when its Options.ReduceMemoryUsage property is true.
See Also
  TJvImageItem.FileName, TJvImageItem.Picture

----------------------------------------------------------------------------------------------------
@@TJvImageItem.Refresh
Summary
  Write here a summary (1 line)
Description
  Write here a description
See Also
  List here other properties, methods (comma seperated)
  Remove the 'See Also' section if there are no references

----------------------------------------------------------------------------------------------------
@@TJvImageLoadErrorEvent
<TITLE TJvImageLoadErrorEvent type>
<COMBINE TJvImagesViewer.OnLoadError>

----------------------------------------------------------------------------------------------------
@@TJvImagesViewer
<TITLEIMG TJvImagesViewer>
JVCLInfo
  GROUP=JVCL.Graphics.Static
  FLAG=Component
Summary
  Control that can display images in a grid like fashion.
Description
  Use the TJvImagesViewer control when you need to display images from disk files or from other sources
  in a grid like control. To display images from disk files, use the Directory and FileMask properties
  to specify what images should be displayed and from where. To display images from other sources, set
  the Count property and use each item's Picture property to assign the images to display.

----------------------------------------------------------------------------------------------------
@@TJvImagesViewer.Directory
Summary
  Specifies the directory to search for files.
Description
  Use the Directory property to specify the directory to search for image files. Directory in
  combination with FileMask defines the file types to find.

  If you set Directory to an empty string, it will be expanded to the current directory (as returned by
  GetCurrentDir).
Note
  Changing Directory triggers the LoadImages method.
See Also
  TJvImagesViewer.FileMask, TJvImagesViewer.LoadImages

----------------------------------------------------------------------------------------------------
@@TJvImagesViewer.FileMask
Summary
  Specifies the file mask to use to filter the files to display.
Description
  Use FileMask to specify the file types to display in the control. FileMask can contain multiple file
  masks and accepts wildcards. Separate individual items with comma or semi-colon. Each individual mask
  item is concatenated with the value of Directory and a search is done on that value.

  The following examples are all valid file masks:
  <PRE>
  *.gif
  *.bmp;*.jpg;*.jpeg
  Jv*.bm?,*.g*,*.ico,*.ani,*.cur 
  </PRE>
Note
  Changing FileMask triggers the LoadImages method.
See Also
  TJvImagesViewer.Directory, TJvImagesViewer.LoadImages

----------------------------------------------------------------------------------------------------
@@TJvImagesViewer.Items
Summary
  Provides indexed access to the items in the control.
Description
  Use Items to access the items in the control.
Parameters
  Index - The index of the item. Valid indices is in the range 0 to Count - 1.
See Also
  Count

----------------------------------------------------------------------------------------------------
@@TJvImagesViewer.LoadImages
Summary
  Updates the control to display the image files available in the selected directory.
Description
  Call LoadImages to reread and display the content of a specific directory.
Return value
  LoadImages returns true if it found at least one matching image.

----------------------------------------------------------------------------------------------------
@@TJvImagesViewer.OnLoadBegin
Summary
  Occurs just before the loading of images starts.
Description
  Write a handler for the OnLoadBegin event to take specific action just before the image loading
  begin. Image loading begins when you call LoadImages or change the Directory or FileMask properties
  and ends when all image files matching the file mask in FileMask located in Directory has been
  loaded.
See Also
  TJvImagesViewer.Directory, TJvImagesViewer.FileMask, TJvImagesViewer.LoadImages,
  TJvImagesViewer.OnLoadEnd

----------------------------------------------------------------------------------------------------
@@TJvImagesViewer.OnLoadEnd
Summary
  Occurs after the loading of images completed.
Description
  Write a handler for the OnLoadEnd event to take specific action just after the image loading ends.
  Image loading begins when you call LoadImages or change the Directory or FileMask properties and ends
  when all image files matching the file mask in FileMask located in Directory has been loaded.
See Also
  TJvImagesViewer.Directory, TJvImagesViewer.FileMask, TJvImagesViewer.LoadImages,
  TJvImagesViewer.OnLoadBegin

----------------------------------------------------------------------------------------------------
@@TJvImagesViewer.OnLoadError
Summary
  Occurs when an error occurs while loading an image.
Description
  Write a handler for the OnLoadError event to take specific actions when an error occurs while loading
  an image from file. For example, if the image file is corrupted or the file does not contain a
  recognized image format, this event is triggered.
Parameters
  Sender   - The object that triggered the event.
  E        - The Exception object that was triggered.
  FileName - The file that triggered the event. This is empty if the image is loaded from some other
              source than a file on disk (like the clipboard or a TPicture object).
  Handled  - Set this parameter to true if you want to handle the error yourself. If Handled is false
              (the default), no error is reported and the item is removed from the Items property. If
              Handled is set to true, the image is cleared and the exception is raised but the item is
              left in the Items property.

----------------------------------------------------------------------------------------------------
@@TJvImagesViewer.OnLoadProgress
Summary
  Occurs periodically during slow operations that affect the image being loaded.
Description
  OnLoadProgress is generated by the particular graphic that the picture object contains. Whether
  OnLoadProgress occurs depends upon the type of graphic the item is loading. Some graphics generate
  this event, others do not. Jpeg images, for example, generate an OnLoadProgress event.

  Write an OnLoadProgress event handler to provide the user with feedback during slow operations such
  as loading large compressed images.
Note
  The PercentDone parameter is only an approximation. With some image formats, the value of PercentDone
  may actually decrease from the value in previous events, as the graphic object discovers there is
  more work to do.
Parameters
  Sender      - The object that triggered the event.
  Item        - The item that is being loaded.
  Stage       - Indicates whether the operation is beginning, continuing, or ending. If the event
                 handler displays an indicator such as a progress bar, the indicator can be created
                 when Stage is psStarting, updated while Stage is psRunning, and removed when Stage is
                 psEnding.
  PercentDone - An approximation of how much of the operation has completed. Use PercentDone to update
                 the position of a progress bar or other indicator.
  RedrawNow   - True if the control should be redrawn.
  R           - Indicates whether the image can safely be drawn on screen at this point. The R
                 parameter specifies the portion of the image that has changed and needs to be
                 redrawn.
  Msg         - Contains one or two words that describe what operation is occurring. For example, the
                 value of Msg could be a string such as Loading, Storing, or Reducing colors. The Msg
                 string can also be empty.

----------------------------------------------------------------------------------------------------
@@TJvImagesViewer.Options
Summary
  Specifies options and default appearance for the images viewer.
Description
  Use the Options property to customize the appearance and functionality of the images viewer.

----------------------------------------------------------------------------------------------------
@@TJvImageViewerLoadProgress
<TITLE TJvImageViewerLoadProgress type>
<COMBINE TJvImagesViewer.OnLoadProgress>

----------------------------------------------------------------------------------------------------
@@TJvImageViewerOptions
Summary
  Class used for the Options property in a TJvImagesViewer.
Description
  TJvImageViewerOptions is the class type for the Options property in a TJvImagesViewer.
  TJvImageViewerOptions derives from TJvCustomItemViewerOptions and publishes many of its properties as
  well as introducing new ones.
See Also
  TJvCustomItemViewerOptions, TJvImagesViewer

----------------------------------------------------------------------------------------------------
@@TJvImageViewerOptions.FrameColor
Summary
  Specifies the color to use for the frame around an item.
Description
  Use FrameColor to set the color used for the frame around an item. If color is set to clNone, no
  frame is displayed.
See Also
  TJvImageViewerOptions.HotColor

----------------------------------------------------------------------------------------------------
@@TJvImageViewerOptions.HotColor
Summary
  Specifies the color to use when an item is hot tracked.
Description
  Use HotColor to specify the color to use when an item i shot tracked. This property only has effect
  when HotTrack is true. If HotColor is set to clNone, no frame is displayed when the mouse hovers over
  the item, regardless of the HotTrack setting.
See Also
  HotTrack

----------------------------------------------------------------------------------------------------
@@TJvImageViewerOptions.HotFrameSize
Summary
  Specifies the size of the frame of a hot tracked item.
Description
  Use HotFrameSize to specify the size of the frame when an item is hot tracked. This property only has
  effect when HotTrack is true,
See Also
  HotTrack

----------------------------------------------------------------------------------------------------
@@TJvImageViewerOptions.ImagePadding
Summary
  Specifies the spacing around the image in an item.
Description
  Use ImagePadding to specify the amount of padding that should surround the image in an item. For
  example, if the item is 20 pixels wide and the ImagePadding is 10, only 10 pixels remain to display
  the image.

----------------------------------------------------------------------------------------------------
@@TJvImageViewerOptions.Transparent
Summary
  Specifies whether the images are drawn transparently.
Description
  Use Transparent to specify whether images are drawn transparently or not.
Note
  Not all image formats supports the notion of transparency.

