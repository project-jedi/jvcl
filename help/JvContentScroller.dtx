##Package: Ctrls
##Status: Completed (I) Rafael Cotta
----------------------------------------------------------------------------------------------------
@@JvContentScroller.pas
Summary
  Contains the TJvContentScroller component.
<INCLUDE JVCL.UnitText.dtx>
Author
  Peter Th�rnqvist

----------------------------------------------------------------------------------------------------
@@TJvContentScrollDirection
<TITLE TJvContentScrollDirection type>
Summary
  Type used to specify the direction of the scrolling.
Description
  TJvContentScrollDirection is the type used to indicate the direction of the scrolling inside the
  control: from bottom to top or from top to bottom.

----------------------------------------------------------------------------------------------------
@@TJvContentScrollDirection.sdUp
  Use sdUp to scroll contents from the bottom to the top.

----------------------------------------------------------------------------------------------------
@@TJvContentScrollDirection.sdDown
  Use sdUp to scroll contests from the top to the bottom.

----------------------------------------------------------------------------------------------------
@@TJvContentScroller
<TITLEIMG TJvContentScroller>
JVCLInfo
  GROUP=JVCL.GridsAndContent.ScrollBoxes
  FLAG=Component
Summary
  A TCustomPanel descendant that can scroll its content.
Description
  Use TJvContentScroller as a container for controls to be scrolled on a specified area (the area of
  the TJvContentScroller).

  You can control the speed and direction of the scrolling, and may specify a wave sound file to be
  played once or continuously while scrolling.

  When the controls inside the TJvContentScroller reach its top (or bottom), they re-appear on the
  TJvContentScroller bottom (or top).

----------------------------------------------------------------------------------------------------
@@TJvContentScroller.Active
Summary
  Starts or stops the scrolling.
Description
  You must set the Active property to true to start the scrolling inside the TJvContentScroller. Set
  Active to false to stop.

  It is possible to set Active to true on design time, and preview how the control will behave in run
  time.

----------------------------------------------------------------------------------------------------
@@TJvContentScroller.LoopCount
Summary
  Determines how many times the will do a loop with its contents.
Description
  Use LoopCount to determine how many times the contents inside the control will loop, that is, how
  many times they will go from the top to the bottom or from the bottom to the top.

  After LoopCount loops, the scrolling stops.

  If you don't want to limit the looping, use LoopCount = -1.

----------------------------------------------------------------------------------------------------
@@TJvContentScroller.LoopMedia
Summary
  Determines if MediaFile plays repedetely or not.
Description
  When set to true, LoopMedia makes the sound specified by MediaFile (if there's one) be played
  repedetely while the contents are scrolled. If LoopMedia is set to false, the sound will be played
  only once.
See Also
  TJvContentScroller.MediaFile

----------------------------------------------------------------------------------------------------
@@TJvContentScroller.MediaFile
Summary
  Specifies a sound file to be played while scrolling.
Description
  Media file will have the name of a .wav file to be played while the scrolling is performed, or only
  once, on the beginning of the scrolling, depending on the LoopMedia value.
See Also
  TJvContentScroller.LoopMedia

----------------------------------------------------------------------------------------------------
@@TJvContentScroller.OnAfterScroll
Summary
  Event called after each scrolling step.
Description
  Write an OnAfterScroll event handler if you want to receive a notification after the execution of
  each scrolling step.
See Also
  TJvContentScroller.OnBeforeScroll

----------------------------------------------------------------------------------------------------
@@TJvContentScroller.OnBeforeScroll
Summary
  Event called before each scrolling step.
Description
  Write an OnBeforeScroll event handler if you want to receive a notification before the execution of
  each scrolling step.
See Also
  TJvContentScroller.OnAfterScroll

----------------------------------------------------------------------------------------------------
@@TJvContentScroller.ScrollAmount
Summary
  Determines in how many pixels the content of a TJvContentScroller will be moved in each step.
Description
  Use the ScrollAmount to determine in how many pixels the content of a TJvContentScroller will be
  moved in each step, up or down, depending on the value of ScrollDirection.

  You can think step as the move performed by the control continuosly, on the interval of
  ScrollIntervall miliseconds.

  The higher is the value of ScrollAmount, the faster is the scrolling.
See Also
  TJvContentScroller.ScrollDirection, TJvContentScroller.ScrollIntervall

----------------------------------------------------------------------------------------------------
@@TJvContentScroller.ScrollContent
Summary
  Function called on each scrolling step that performs the scrolling.
Description
  ScrollContent is the function called inside the control's timer that performs the scrolling.
Parameters
  Amount - Determines in how many pixels the controls will be moved by.
See Also
  TJvContentScroller.ScrollAmount

----------------------------------------------------------------------------------------------------
@@TJvContentScroller.ScrollDirection
Summary
  Determines the direction of the scrolling.
Description
  Use ScrollDirection to determine whether the scrolling will be done from the bottom to top, setting
  this to sdUp, or from top to bottom, setting this to sdDown.

----------------------------------------------------------------------------------------------------
@@TJvContentScroller.ScrollIntervall
Summary
  Determines the time elapsed between each step on the scrolling.
Description
  The ScrollIntervall controls the interval between each step on scrolling, that is, the time elapsed
  each time the content is scrolled ScrollAmount pixels up or down.

  The lower if the value for ScrollIntervall, the faster and smoother will be the scrolling, but it
  will also consume more CPU cycles.

----------------------------------------------------------------------------------------------------
@@TJvContentScroller.ScrollLength
Summary
  Determines the range of the scrolling.
Description
  Use ScrollLength to determine the range of the scrolling.

  You'll probably prefer a value close to the Height value. Smaller values will make the controls
  inside the TJvContentScroller disapear before reaching the top/bottom border of the control.

  High values will create a delay effect, that is, the contens will reach the border, and will take
  some time to reappear on the opposite border.

----------------------------------------------------------------------------------------------------
@@TJvScrollAmount
<TITLE TJvScrollAmount type>
Summary
  Used TJvScrollAmount to specify the values of ScrollAmount, ScrollIntervall and ScrollLength.
Description
  TJvScrollAmount ranges from 1 to 65535.

