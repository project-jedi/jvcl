##Package: Ctrls
##Status: Completed (I) (peter3) [Note case diff: TJvRegistryTreeView <> JvRegistryTreeview.pas (rb)]
##Skip: TJvRegistryTreeView.ShowHint
----------------------------------------------------------------------------------------------------
@@JvRegistryTreeview.pas
Summary
  Contains the TJvRegistryTreeView component.
<INCLUDE JVCL.UnitText.dtx>
Author
  Peter Th�rnqvist


----------------------------------------------------------------------------------------------------
@@TJvRegistryTreeView
<TITLEIMG TJvRegistryTreeView>
JVCLInfo
  GROUP=JVCL.ListsAndTrees.Trees
  FLAG=Component
Summary
  Treeview that displays the hives and subnodes of the registry.
Description
  Use a TJvRegistryTreeview as a simple way of displaying the content of the registry with no coding
  necessary.

----------------------------------------------------------------------------------------------------
@@TJvRegistryTreeView.AddBinaryValue
Summary
  Adds a binary value at the specified location.
Description
  Use AddBinaryValue to add a new binary value to the specified key.
Parameters
  ParentNode - The node that should contain the new value.
  Name       - The name of the new value.
  Buf        - The data to save.
  BufSize    - The size of the data.
Return value
  Returns the newly created node or nil if no node was created.
See Also
  <LINK AddDWORDValue>, <LINK AddStringValue>, <LINK AddKey>

----------------------------------------------------------------------------------------------------
@@TJvRegistryTreeView.AddDWORDValue
Summary
  Adds a DWORD value at the specified location.
Description
  Use AddDWORDValue to add a new or update an existing DWORD value in the specified key.
Parameters
  ParentNode - The node that should contain the new value.
  Name       - The name of the new or existing value.
  Value      - The value.
Return value
  Returns the newly created node or nil if no node was created.
See Also
  <LINK AddBinaryValue>, <LINK AddStringValue>, <LINK AddKey>

----------------------------------------------------------------------------------------------------
@@TJvRegistryTreeView.AddKey
Summary
  Adds a new key at the specified location.
Description
  Use AddKey to add a new key to the registry. The key is added as the child of the specified node.
Parameters
  ParentNode - The parent of the new key.
  KeyName    - The name of the key to create.
Return value
  Returns the newly created node or nil if no node was created.
See Also
  <LINK AddBinaryValue>, <LINK AddDWORDValue>, <LINK AddStringValue>

----------------------------------------------------------------------------------------------------
@@TJvRegistryTreeView.AddStringValue
Summary
  Adds a string value at the specified location.
Description
  Use AddStringValue to add a new or update an existing string value in the specified key.
Parameters
  ParentNode - The node that should contain the new value.
  Name       - The name of the value.
  Value      - The value.
Return value
  Returns the newly created node or nil if no node was created.
See Also
  <LINK AddBinaryValue>, <LINK AddDWORDValue>, <LINK AddKey>

----------------------------------------------------------------------------------------------------
@@TJvRegistryTreeView.CurrentKey
Summary
  Specifies the currently select key (node) in the treeview.
Description
  \Read CurrentKey to access the HKEY value of the currently selected key. You can use the returned
  value as a parameter to Windows API functions that expect a HKEY as parameter.
See Also
  TJvRegistryTreeView.CurrentPath, TJvRegistryTreeView.ShortPath

----------------------------------------------------------------------------------------------------
@@TJvRegistryTreeView.CurrentPath
Summary
  Specifies the path to the currently selected key including the "My Computer" root item.
Description
  \Read CurrentPath to get the full path to the currently selected node in the tree, including the
  text on the "My Computer" root item. Nodes are delimited by '\' characters.
See Also
  TJvRegistryTreeView.CurrentKey, TJvRegistryTreeView.RootCaption, TJvRegistryTreeView.ShortPath

----------------------------------------------------------------------------------------------------
@@TJvRegistryTreeView.DefaultCaption
Summary
  Specifies the caption for items with empty names (default items) when displayed in the ListView.
Description
  Use DefaultCaption to change the text displayed in the ListView when an item doesn't have a name of
  its own. Registry keys always have one, unnamed, item and this property is used to set a display
  name for it.
See Also
  TJvRegistryTreeView.DefaultNoValueCaption

----------------------------------------------------------------------------------------------------
@@TJvRegistryTreeView.DefaultNoValueCaption
Summary
  Specifies the text to display when an unnamed item doesn't have a value.
Description
  Use DefaultNoValueCaption to set the text to display in ListView when an unnamed item doesn't have
  a value. This string is only used when the value of the unnamed item is empty.
See Also
  TJvRegistryTreeView.DefaultCaption

----------------------------------------------------------------------------------------------------
@@TJvRegistryTreeView.Items
Summary
  Lists the individual nodes that appear in the tree view control.
Description
  Individual nodes in a tree view are TTreeNode objects. These individual nodes can be accessed by using the Items property along with th
   item's index into the tree view. The Items property if this component should be considered as
  read-only although you can add new nodes in code.
Note
  Adding new nodes using Items, will not affect the registry. To add new keys and values to the
  registry use the functions provided specifically by the TJvRegistryTreeview.
See Also
  <LINK AddKey>, <LINK AddBinaryValue>, <LINK AddDWORDValue>, <LINK AddStringValue>

----------------------------------------------------------------------------------------------------
@@TJvRegistryTreeView.ListView
Summary
  Specifies a listview used to display the content of the selected node.
Description
  Assign a listview to the ListView property to have it automatically populated when the selected
  node in the treeview changes. If you display the items in report mode, add columns to the listview
  with the following names: "\Name", "\Value", "\Type". The "\Type" column doesn't have to be added
  if you don't want to display the data type for items. If the listview is assigned, some items are
  given the text values specified by DefaultCaption and DefaultNoValueCaption.
See Also
  TJvRegistryTreeView.DefaultCaption, TJvRegistryTreeView.DefaultNoValueCaption

----------------------------------------------------------------------------------------------------
@@TJvRegistryTreeView.LoadKey
Summary
  Creates a subkey under the root key and loads registry information from a file into the newly
  created subkey.
Description
  Call LoadKey to: 1.  \Create a new subkey under the root key, and 2.  Load registry information
  from a file into the subkey. Registry information can include data      values, subkeys, and data
  values for those subkeys.
  
  LoadKey is intended to simplify creation of a key, its values and subkeys, and the values for those
  subkeys in a single operation. A key, its subkeys, and all data values of the key and its
  subkeys is called a hive. Rather than creating each key and value separately, an application can
  read a hive from a file. This is especially useful for applications that users can reconfigure at
  run time.
Parameters
  Filename - The location of the file containing registry information to store in the subkey. The
  file             specified by FileName must be one previously created using the <LINK SaveKey>
  function or the
              RegSaveKey Windows API function. On systems that use a file allocation table (FAT),
  FileName
              cannot include an extension.
Return value
  Returns true if the data was successfully loaded.
Note
  The key to load data into is determined by the CurrentKey property.
See Also
  <LINK SaveKey>

----------------------------------------------------------------------------------------------------
@@TJvRegistryTreeView.RefreshNode
Summary
  Updates the node's data.
Description
  Call RefreshNode to refresh the display for the selected node. To re-read all data from the
  registry, set Node to nil. Calling RefreshNode will:   1. \Delete the subnodes in the treeview, and
    2. Re-read all the data from the registry again
Parameters
  Node - The node to refresh

----------------------------------------------------------------------------------------------------
@@TJvRegistryTreeView.RegistryKeys
Summary
  Specifies the registry keys ("hives") to display in the tree.
Description
  Set RegistryKeys to define what nodes are displayed in the treeview. If RegistryKeys is set to [],
  nothing will be displayed. The default is to display HKEY_LOCAL_MACHINE and HKEY_LOCAL_USER.
Note
  Not all available values are available on all versions of Windows. If a specific hive is not found,
  it is silently skipped.

----------------------------------------------------------------------------------------------------
@@TJvRegistryTreeView.RootCaption
Summary
  Specifies the text to display for the root item in the treeview.
Description
  The root of the treeview displays a "My Computer" icon. This node is not an actual node in the
  registry but is provided to emulate the way the regedit.exe application displays the registry.
  Use RootCaption to change the text displayed for this node. By default, the text is set to "My
  Computer".

----------------------------------------------------------------------------------------------------
@@TJvRegistryTreeView.SaveKey
Summary
  Opens the specified key with the security access value KEY_ALL_ACCESS and saves the specified key
  and all of its subkeys and values to a hive file.
Description
  Call SaveKey to open a key with a security access value of KEY_ALL_ACCESS, and save the key and its
  subkeys and data values to a hive file. A hive is a discrete collection of keys, subkeys, and
  values that is rooted at the top of the registry hierarchy.
  
  The key to save is read from the CurrentKey property.
  
  Files created by SaveKey can be passed to the <LINK LoadKey> function.
Parameters
  Filename - the file into which to save the key information. It must be the name of a new file that 
             does not already exist. On FAT file systems FileName cannot include an extension.
Return value
  If SaveKey is successful it returns True.
See Also
  <LINK LoadKey>

----------------------------------------------------------------------------------------------------
@@TJvRegistryTreeView.ShortPath
Summary
  Returns the path to the current node without including the "My Computer" root node.
Description
  Use ShortPath when you need to retrieve a path string to the currently selected node. ShortPath is
  just like CurrentPath but does not include the value of the "My Computer" item in the string.
See Also
  TJvRegistryTreeView.CurrentPath

