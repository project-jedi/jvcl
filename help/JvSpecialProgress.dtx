----------------------------------------------------------------------------------------------------
@@$TJvSpecialProgress
<TITLE TJvSpecialProgress overview>
<GROUP JVCL.Miscel>
  TJvSpecialProgress is a progress bar, filled using a two-color gradient fill. That is, the color 
  of the progress bar turns from left to right gradually from StartColor into EndColor.
  
  Set TextVisible to True, to display the Position as percentage of (Maximum - Minimum). Use TextCentered
  and TextFont to alter the appearance of the percentage text. 
  
  Use Color to change the background color of the control, or set Transparant to True to set the 
  background color to transparant.
----------------------------------------------------------------------------------------------------
@@JvSpecialProgress.pas
Summary
  Contains the JvSpecialProgress component.
<INCLUDE JVCL.UnitText.dtx>
Author
  S�bastien Buysse
----------------------------------------------------------------------------------------------------
@@TJvSpecialProgress
<GROUP TJvSpecialProgress>
Summary
  Two-color gradient filled progress bar.
Description
  TJvSpecialProgress is a progress bar, filled using a two-color gradient fill. That is, the color 
  of the progress bar turns from left to right gradually from StartColor into EndColor.
  
  Set TextVisible to True, to display the Position as percentage of (Maximum - Minimum). Use TextCentered
  and TextFont to alter the appearance of the percentage text. 
  
  Use Color to change the background color of the control, or set Transparant to True to set the 
  background color to transparant.
----------------------------------------------------------------------------------------------------
@@TJvSpecialProgress.Create
  Creates and initializes an instance of TJvSpecialProgress.
Description
  Call Create to create an instance of TJvSpecialProgress at runtime. A TJvSpecialProgress placed on a 
  form at design time is created automatically.

  Pass a single Component as a parameter to provide the progress bar with an Owner that is responsible
  for freeing it.

  Create calls the inherited Create method, creates a timer, and then sets the initial values for the 
  progress bar as follows: 
  
  * Minimum is set to 0.
  * Maximum is set to 100.
  * StartColor is set to clWhite, EndColor is set to clBlack.
  * Position is set to 0.
  * Solid, TextCentered, TextVisible and Transparent are set to False.
Parameters
  AOwner - The owner for the instance of TJvSpecialProgress.
See Also
  Destroy, EndColor, Maximum, Minimum, Solid, StartColor, TextCentered, TextVisible, Transparent, 
----------------------------------------------------------------------------------------------------
@@TJvSpecialProgress.Destroy
Summary
  Destroys an instance of TJvSpecialProgress.
Description
  Do not call Destroy directly in an application. Instead, call Free. Free verifies that the 
  TJvSpecialProgress object is not nil and only then calls Destroy.
Parameters
  None.
See Also
  Create
----------------------------------------------------------------------------------------------------
@@TJvSpecialProgress.AboutJVCL
  <INCLUDE JVCL.Main.AboutJVCL.dtx>
----------------------------------------------------------------------------------------------------
@@TJvSpecialProgress.Color
Summary
  Specifies the background color of the control.
Description
  Use Color to read or change the background color of the control.
See Also
  EndColor, StartColor, Transparent
----------------------------------------------------------------------------------------------------
@@TJvSpecialProgress.EndColor
Summary
  Determines the end color for the gradient fill used to fill the progress bar.
Description
  Use EndColor to specify the end color for the gradient fill. The progress bar is filled using a 
  two-color gradient fill. That is, the color of the progress bar turns from left to right gradually 
  from StartColor into EndColor.
See Also
  StartColor
----------------------------------------------------------------------------------------------------
@@TJvSpecialProgress.HintColor
Summary
  Determines the color of the hint boxes for the Help Hints for the progress bar.
Description
  Use HintColor to specify the hint box color. A default color value of clInfoBk is set for the
  HintColor property in the constructor when the application is created. To change the HintColor
  assign it a new TColor value.
See Also
----------------------------------------------------------------------------------------------------
@@TJvSpecialProgress.Maximum
Summary
  Specifies the upper limit of the range of possible positions.
Description
  Use Maximum along with the Minimum property to establish the range of possible positions for a progress
  bar. When the process tracked by the progress bar is complete, the value of Position should equal 
  Maximum.
See Also
  Minimum, Position
----------------------------------------------------------------------------------------------------
@@TJvSpecialProgress.Minimum
Summary
  Specifies the lower limit of the range of possible positions.
Description
  Use Maximum along with the Minimum property to establish the range of possible positions for a progress 
  bar. When the process tracked by the progress bar begins, the value of Position should equal Minimum.
See Also
  Maximum, Position
----------------------------------------------------------------------------------------------------
@@TJvSpecialProgress.OnMouseEnter
Summary
  Occurs when the mouse pointer moves over the scroll box.
Description
  Write an OnMouseEnter event handler to take specific action when the user moves the mouse over the
  progress bar. For example, you can use this event to change the font color when the mouse is over the
  progress bar, and then use the OnMouseLeave event to change it back when the mouse moves off the 
  scroll box.
See Also
  OnMouseLeave 
----------------------------------------------------------------------------------------------------
@@TJvSpecialProgress.OnMouseLeave
Summary
  Occurs when the mouse pointer moves off the scroll box.
Description
  Write an OnMouseLeave event handler to take specific action when the user moves the mouse off the
  progress bar. For example, you can use this event to undo changes that were made in an OnMouseEnter
  event handler.
See Also
  OnMouseEnter
----------------------------------------------------------------------------------------------------
@@TJvSpecialProgress.OnParentColorChange
Summary
  Occurs when the color property for the parent of the scroll box changes.
Description
  Write an OnParentColorChange event to take specific action when the ParentColor property changes.
----------------------------------------------------------------------------------------------------
@@TJvSpecialProgress.Paint
Summary
  Renders the image of the sizeable panel.
Description
  Paint is called automatically in response to WM_PAINT messages. TJvSpecialProgress overrides Paint
  in order to render the gradient filled progress bar.
Parameters
  None.
----------------------------------------------------------------------------------------------------
@@TJvSpecialProgress.Position
Summary
  Specifies the current position of the progress bar.
Description
  You can read Position to determine how far the process tracked by the progress bar has advanced 
  from Minimum toward Maximum. Set Position to cause the progress bar to display a position between 
  Minimum and Maximum. For example, when the process tracked by the progress bar completes, set 
  Position to Maximum so that it appears completely filled.

  When a progress bar is created, Minimum and Maximum represent percentages, where Minimum is 0 
  (0% complete) and Maximum is 100 (100% complete). If these values are not changed, Position is 
  the percentage of the process that has already been completed.
See Also
  Maximum, Minimum
----------------------------------------------------------------------------------------------------
@@TJvSpecialProgress.Solid
Summary
  Specifies whether the progress bar is solid or segmented.
Description
  Use Solid to specify whether the progress bar is solid or segmented.
----------------------------------------------------------------------------------------------------
@@TJvSpecialProgress.StartColor
Summary
  Determines the start color for the gradient fill used to fill the progress bar.
Description
  Use StartColor to specify the start color for the gradient fill. The progress bar is filled using a 
  two-color gradient fill. That is, the color of the progress bar turns from left to right gradually 
  from StartColor into EndColor.
See Also
  EndColor
----------------------------------------------------------------------------------------------------
@@TJvSpecialProgress.Step
Summary
  Specifies the amount that Position increases when the StepIt method is called.
Description
  Set Step to specify the granularity of the progress bar. Step should reflect the size of each step
  in the process tracked by the progress bar, in the logical units used by the Maximum and Minimum 
  properties. 

  When a progress bar is created, Minimum and Maximum represent percentages, where Minimum is 0 
  (0% complete) and Maximum is 100 (100% complete). If these values are not changed, Step is the 
  percentage of the process completed before the user is provided with additional visual feedback.

  When the StepIt method is called, the value of Position increases by Step.
See Also
  Maximum, Minimum, Position, StepIt
----------------------------------------------------------------------------------------------------
@@TJvSpecialProgress.StepIt
Summary
  Advances Position by the amount specified in the Step property.
Description
  Call the StepIt method to increase the value of Position by the value of the Step property. If Step 
  represents the size of one logical step in the process tracked by the progress bar, call Step after 
  each logical step is completed.
Parameters
  None.
See Also
  Step
----------------------------------------------------------------------------------------------------
@@TJvSpecialProgress.TextCentered
Summary
  Determines the placement of the text on the progress bar.
Description
  Set TextCentered to True to keep the position of the text unchanged while changing property Position.
  That is, centered on the control. Set TextCentered to False to position the text centered on the
  filled part of the progress bar. If property Position changes, then the progress bar size will 
  change and thus, so will the position of the text. 
  
  TextCentered must be set to True to display the text on the progress bar.
See Also
   TextFont, TextVisible
----------------------------------------------------------------------------------------------------
@@TJvSpecialProgress.TextFont
Summary
  Controls the attributes of text written on the progress bar.
Description
  To change to a new font, specify a new TFont object. To modify a font, change the value of the 
  Charset, Color, Height, Name, Pitch, Size, or Style of the TFont object.
  
  TextVisible must be set to True to display the text on the progress bar.
See Also
  TextCentered, TextVisible
----------------------------------------------------------------------------------------------------
@@TJvSpecialProgress.TextVisible
Summary
  Determines whether text is displayed on the progress bar.
Description
  If TextVisible is True, value Position as percentage of (Maximum - Minimum) is displayed on the 
  progress bar. For example, if Maximum = 4, Minimum = 0 and Position = 3 then the display text will
  be '75%'. If TextVisible is False, no text is displayed.
  
  You cannot customize the displayed text, except by changing Maximum, Minimum or Position.
See Also
 Maximum, Minimum, Position, TextCentered, TextFont
----------------------------------------------------------------------------------------------------
@@TJvSpecialProgress.Transparent
Summary
  Specifies whether the background of the progress bar is transparent.
Description
  Use Transparent to specify whether the background of the progress bar is transparent.
See Also
  Color