##Package: MM
##Status: Completed (I)
----------------------------------------------------------------------------------------------------
@@JvSpecialProgress.pas
Summary
  Contains the TJvSpecialProgress component.
<INCLUDE JVCL.UnitText.dtx>
Author
  S�bastien Buysse
----------------------------------------------------------------------------------------------------
@@TJvSpecialProgress
<TITLEIMG TJvSpecialProgress>
JVCLInfo
  GROUP=JVCL.Bars.SlideAndProgress.Progress
  FLAG=Component
Summary
  Two-color gradient filled progress bar.
Description
  TJvSpecialProgress is a progress bar, filled using a two-color gradient fill. That is, the color
  of the progress bar turns from left to right gradually from StartColor into EndColor.

  Use TextOption, to display the current progress percentage of the process tracked by the
  progress bar, or display a custom string. \Read PercentDone to get this value. Use TextCentered
  and Font to alter the appearance of the percentage text.

  Use Color to change the background color of the control, or set Transparant to true to set the
  background color to transparant.
----------------------------------------------------------------------------------------------------
@@TJvSpecialProgress.BorderStyle
Summary
  Determines whether the progress bar has a border.
Description
  Use BorderStyle to change the appearance of the outer boundary of the progress bar control. A
  progress bar can have a single border (bsSingle) or no visible border (bsNone).
----------------------------------------------------------------------------------------------------
@@TJvSpecialProgress.Color
Summary
  Specifies the background color of the control.
Description
  Use Color to read or change the background color of the control.
See Also
  EndColor, StartColor
----------------------------------------------------------------------------------------------------
@@TJvSpecialProgress.Create
Summary
  Creates and initializes an instance of TJvSpecialProgress.
Description
  Call \Create to create an instance of TJvSpecialProgress at runtime. A TJvSpecialProgress placed on a
  form at design time is created automatically.

  Pass a single Component as a parameter to provide the progress bar with an Owner that is responsible
  for freeing it.

  \Create calls the inherited \Create method, creates a timer, and then sets the initial values for the
  progress bar as follows:

  * Minimum is set to 0.
  * Maximum is set to 100.
  * StartColor is set to clWhite, EndColor is set to clBlack.
  * Position is set to 0.
  * Solid, TextCentered, GradientBlocks and Transparent are set to false.
Parameters
  AOwner - The owner for the instance of TJvSpecialProgress.
See Also
  Destroy, EndColor, GradientBlocks, Maximum, Minimum, Solid, StartColor,
  TextCentered, TextOption
----------------------------------------------------------------------------------------------------
@@TJvSpecialProgress.Destroy
Summary
  Destroys an instance of TJvSpecialProgress.
Description
  Do not call Destroy directly in an application. Instead, call Free. Free verifies that the
  TJvSpecialProgress object is not nil and only then calls Destroy.
See Also
  Create
----------------------------------------------------------------------------------------------------
@@TJvSpecialProgress.EndColor
Summary
  Determines the end color for the gradient fill used to fill the progress bar.
Description
  Use EndColor to specify the end color for the gradient fill. The progress bar is filled using a
  two-color gradient fill. That is, the color of the progress bar turns from left to right gradually
  from StartColor into EndColor.
See Also
  StartColor
----------------------------------------------------------------------------------------------------
@@TJvSpecialProgress.Font
Summary
  \Controls the attributes of text written on the progress bar.
Description
  To change to a new font, specify a new TFont object. To modify a font, change the value of the
  Charset, Color, Height, Name, Pitch, Size, or Style of the TFont object.

  TextOption must not be set to true to display the text on the progress bar.
See Also
  TextCentered, TextOption
----------------------------------------------------------------------------------------------------
@@TJvSpecialProgress.GradientBlocks
Summary
  Specifies the fill method of the progress blocks.
Description
  Set GradientBlocks to false to let the progress control fill each progress block using a one color fill.
  From left to right, the fill color turns gradually from StartColor into EndColor. Set GradientBlocks
  to true to let the control fill each progress blocks using a two-color gradient fill. From left
  to right, the begin color and end color of each block turns gradually from StartColor to EndColor.
Note
  Property Solid must be set to false, otherwise property GradientBlocks is ignored.
See Also
  Solid
----------------------------------------------------------------------------------------------------
@@TJvSpecialProgress.HintColor
Summary
  Determines the color of the hint boxes for the Help Hints for the progress bar.
Description
  Use HintColor to specify the hint box color. A default color value of clInfoBk is set for the
  HintColor property in the constructor when the application is created. To change the HintColor
  assign it a new TColor value.
----------------------------------------------------------------------------------------------------
@@TJvSpecialProgress.Loaded
Summary
  Initializes the progress bar after its form is loaded into memory.
Description
  Do not call the protected Loaded method. The component streaming system calls this method after it
  loads the progress bar�s form from a stream.

  TJvSpecialProgress overrides the inherited method to \paint the in-memory bitmap.
----------------------------------------------------------------------------------------------------
@@TJvSpecialProgress.Maximum
Summary
  Specifies the upper limit of the range of possible positions.
Description
  Use Maximum along with the Minimum property to establish the range of possible positions for a progress
  bar. When the process tracked by the progress bar is complete, the value of Position should equal
  Maximum.
See Also
  Minimum, Position
----------------------------------------------------------------------------------------------------
@@TJvSpecialProgress.Minimum
Summary
  Specifies the lower limit of the range of possible positions.
Description
  Use Minimum along with the Maximum property to establish the range of possible positions for a progress
  bar. When the process tracked by the progress bar begins, the value of Position should equal Minimum.
See Also
  Maximum, Position
----------------------------------------------------------------------------------------------------
@@TJvSpecialProgress.OnParentColorChange
Summary
  Occurs when the color property for the parent of the scroll box changes.
Description
  \Write an OnParentColorChange event to take specific action when the ParentColor property changes.
----------------------------------------------------------------------------------------------------
@@TJvSpecialProgress.Paint
Summary
  Renders the image of the sizeable panel.
Description
  Paint is called automatically in response to WM_PAINT messages. TJvSpecialProgress overrides Paint
  in order to render the gradient filled progress bar.
----------------------------------------------------------------------------------------------------
@@TJvSpecialProgress.PercentDone
Summary
  Indicates the current progress percentage of the process tracked by the progress bar.
Description
  \Read PercentDone to get the current progress percentage of the process tracked by the progress bar.
  The value equals (Position - Minimum) as a percentage of (Maximum - Minimum).
See Also
  Maximum, Minimum, Position
----------------------------------------------------------------------------------------------------
@@TJvSpecialProgress.Position
Summary
  Specifies the current position of the progress bar.
Description
  You can read Position to determine how far the process tracked by the progress bar has advanced
  from Minimum toward Maximum. Set Position to cause the progress bar to display a position between
  Minimum and Maximum. For example, when the process tracked by the progress bar completes, set
  Position to Maximum so that it appears completely filled.

  When a progress bar is created, Minimum and Maximum represent percentages, where Minimum is 0
  (0% complete) and Maximum is 100 (100% complete). If these values are not changed, Position is
  the percentage of the process that has already been completed.
See Also
  Maximum, Minimum
----------------------------------------------------------------------------------------------------
@@TJvSpecialProgress.Solid
Summary
  Specifies whether the progress bar is solid or segmented.
Description
  Use Solid to specify whether the progress bar is solid or segmented. If Solid is set to true, the
  progres bar is solid and the GradientBlocks property is ignored. If Solid is set to false, the progress
  bar is segmented. GradientBlocks then indicates whether the progress blocks are filled using a
  two-color gradient fill or using one color.
See Also
  GradientBlocks
----------------------------------------------------------------------------------------------------
@@TJvSpecialProgress.StartColor
Summary
  Determines the start color for the gradient fill used to fill the progress bar.
Description
  Use StartColor to specify the start color for the gradient fill. The progress bar is filled using a
  two-color gradient fill. That is, the color of the progress bar turns from left to right gradually
  from StartColor into EndColor.
See Also
  EndColor
----------------------------------------------------------------------------------------------------
@@TJvSpecialProgress.Step
Summary
  Specifies the amount that Position increases when the StepIt method is called.
Description
  Set Step to specify the granularity of the progress bar. Step should reflect the size of each step
  in the process tracked by the progress bar, in the logical units used by the Maximum and Minimum
  properties.

  When a progress bar is created, Minimum and Maximum represent percentages, where Minimum is 0
  (0% complete) and Maximum is 100 (100% complete). If these values are not changed, Step is the
  percentage of the process completed before the user is provided with additional visual feedback.

  When the StepIt method is called, the value of Position increases by Step.
See Also
  Maximum, Minimum, Position, StepIt
----------------------------------------------------------------------------------------------------
@@TJvSpecialProgress.StepIt
Summary
  Advances Position by the amount specified in the Step property.
Description
  Call the StepIt method to increase the value of Position by the value of the Step property. If Step
  represents the size of one logical step in the process tracked by the progress bar, call Step after
  each logical step is completed.
See Also
  Step
----------------------------------------------------------------------------------------------------
@@TJvSpecialProgress.TextCentered
Summary
  Determines the placement of the text on the progress bar.
Description
  Set TextCentered to true to keep the position of the text unchanged - that is, centered on the
  control - while changing property Position. Set TextCentered to false to position the text centered on the
  filled part of the progress bar. If property Position changes, then the progress bar size will
  change and thus, so will the position of the text.
Note
  TextOption must be set to a value other than toNoText, to display the text on the progress bar.
See Also
   Font, TextOption
----------------------------------------------------------------------------------------------------
@@TJvSpecialProgress.TextOption
Summary
  Determines whether or which text is displayed on the progress bar.
Description
  Use TextOption to specify to:

  * Not display text on the progress bar, or
  * Display a percentage, or
  * Display the text specified by the Caption property, or
  * Display a formatted text.
See Also
  PercentDone, Position, TextCentered, Font, TJvTextOption
----------------------------------------------------------------------------------------------------
@@TJvSpecialProgress.UpdateBuffer
Summary
  Updates the in-memory bitmap of the progress bar.
Description
  TJvSpecialProgress uses double buffering to reduces the amount of flicker when the control repaints.
  The control paints itself to an in-memory bitmap that is then used to \paint the window.
  UpdateBuffer is called automatically when the in-memory bitmap needs to be updated.
See Also
  UpdateTaille
----------------------------------------------------------------------------------------------------
@@TJvTextOption
<TITLE TJvTextOption type>
Summary
  Represents different kinds of display text options for a progress bar.
Description
  Use TJvTextOption values to specify what text to display on a progress bar.
@@TJvTextOption.toCaption
  Displays the text specified by the Caption property on the progressbar.
@@TJvTextOption.toFormat
  Displays a formatted text on the progressbar. Ensure that the Caption property of the progress bar
  contains a '\%s', for example set Caption to 'Progress is \%s%%'. When
  <LINK TJvSpecialProgress.PercentDone,PercentDone> equals 10 the
  displayed text will be 'Progress is 10\%'.
@@TJvTextOption.toNoText
  Displays no text on the progressbar.
@@TJvTextOption.toPercent
  Displays the percentage done on the progressbar.
