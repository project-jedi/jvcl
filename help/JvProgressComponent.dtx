##Package: Dlgs
##Status: Incomplete
----------------------------------------------------------------------------------------------------
@@JvProgressComponent.pas
Summary
  Contains the TJvProgressComponent component.
<INCLUDE JVCL.UnitText.dtx>
Author
  Andrei Prygounkov


----------------------------------------------------------------------------------------------------
@@TJvProgressComponent
<TITLEIMG TJvProgressComponent>
JVCLInfo
  GROUP=JVCL.Bars.SlideAndProgress.Progress,JVCL.Dialogs.Custom
  FLAG=Component
Summary
  A component that can display a progress dialog.
Description
  Use a TJvProgressComponent when you need to display a progress dialog in your application. Progress
  dialogs are mostly used to provide feedback to the user when the application is performing a
  lengthy process and the program needs to be disabled until finished.
  
  To display the dialog, call the Execute method. While the dialog is visible, you can modify any of
  its properties.

----------------------------------------------------------------------------------------------------
@@TJvProgressComponent.Cancel
Summary
  Specifies if the user clicked the Cancel button to close the dialog.
Description
  Read Cancel to determine if the dialog was closed because the user clicked the Cancel button. The
  value of
  Cancel is always false while the dialog is displayed.
See Also
  Execute, TJvProgressComponent.OnShow

----------------------------------------------------------------------------------------------------
@@TJvProgressComponent.Caption
Summary
  Specifies the text displayed on the dialogs caption.
Description
  Use Caption to change the Caption of the dialog form. The value can be changed even when the dialog
  is displayed.
See Also
  TJvProgressComponent.InfoLabel

----------------------------------------------------------------------------------------------------
@@TJvProgressComponent.Execute
Summary
  Displays the progress dialog.
Description
  Call Execute to display the dialog. Before calling Execute, adjust the Caption, InfoLabel,
  ProgressMax, ProgressMin and ProgressPosition properties to their initial values. While the dialog
  is displayed, modify the properties of the component to update the values in the dialog.
See Also
  TJvProgressComponent.Caption, TJvProgressComponent.InfoLabel, TJvProgressComponent.ProgressMax,
  TJvProgressComponent.ProgressMin, TJvProgressComponent.ProgressPosition

----------------------------------------------------------------------------------------------------
@@TJvProgressComponent.Hide
Summary
  Write here a summary (1 line)
Description
  Write here a description
See Also
  List here other properties, methods (comma seperated)
  Remove the 'See Also' section if there are no references

----------------------------------------------------------------------------------------------------
@@TJvProgressComponent.InfoLabel
Summary
  Specifies the text displayed above the progress bar.
Description
  Use InfoLabel to display a descriptive text above the progress bar on the dialog.
See Also
  TJvProgressComponent.Caption

----------------------------------------------------------------------------------------------------
@@TJvProgressComponent.OnShow
Summary
  Occurs when the dialog is displayed.
Description
  Write a handler for the OnShow event to take specific action when the dialog is displayed. For
  example, use this event to reset the dialogs properties to their initial values. OnShow is
  triggered as a result of calling Execute.
See Also
  Execute

----------------------------------------------------------------------------------------------------
@@TJvProgressComponent.ProgressMax
Summary
  Specifies the upper limit of the range of possible positions for the progress bar.
Description
  Use ProgressMax along with the \ProgressMin property to establish the range of possible positions
  for the progress bar.
  Optimally, when the process tracked by the progress bar is complete, the value of \ProgressPosition
  should equal ProgressMax.
See Also
  TJvProgressComponent.ProgressMin, TJvProgressComponent.ProgressPosition,
  TJvProgressComponent.ProgressStepIt

----------------------------------------------------------------------------------------------------
@@TJvProgressComponent.ProgressMin
Summary
  Specifies the lower limit of the range of possible positions for the progress bar.
Description
  Use \ProgressMax along with the ProgressMin property to establish the range of possible positions
  for the progress bar.
  Optimally, when the process tracked by the progress bar is complete, the value of \ProgressPosition
  should equal ProgressMax.
See Also
  TJvProgressComponent.ProgressMax, TJvProgressComponent.ProgressPosition,
  TJvProgressComponent.ProgressStep, TJvProgressComponent.ProgressStepIt

----------------------------------------------------------------------------------------------------
@@TJvProgressComponent.ProgressPosition
Summary
  Specifies the current position of the progress bar.
Description
  Read ProgressPosition to determine how far the process tracked by the progress bar has advanced
  from \ProgressMin toward \ProgressMax.
  Set ProgressPosition to cause the progress bar to display a position between \ProgressMin and 
  ProgressMax.
  For example, when the process tracked by the progress bar completes, set ProgressPosition to 
  ProgressMax so that it appears completely filled.
  
  When a progress bar is created, \ProgressMin and \ProgressMax represent percentages,
  where \ProgressMin is 0 (0% complete) and \ProgressMax is 100 (100% complete). If these values are
  not changed,
  ProgressPosition is the percentage of the process that has already been completed.
See Also
  TJvProgressComponent.ProgressMax, TJvProgressComponent.ProgressMin,
  TJvProgressComponent.ProgressStep, TJvProgressComponent.ProgressStepIt

----------------------------------------------------------------------------------------------------
@@TJvProgressComponent.ProgressStep
Summary
  Specifies the amount that \ProgressPosition increases when the \ProgressStepIt method is called.
Description
  Set ProgressStep to specify the granularity of the progress bar. ProgressStep should reflect the
  size of each step in the process tracked by the progress bar, in
  the logical units used by the \ProgressMax and \ProgressMin properties.
  
  When a progress bar is created, \ProgressMin and \ProgressMax represent
  percentages, where \ProgressMin is 0 (0% complete) and \ProgressMax is 100 (100% complete).
  If these values are not changed, ProgressStep is the percentage of the process completed before the
  user is provided with additional visual feedback.
  
  When the ProgressStepIt method is called, the value of \ProgressPosition increases by Step.
See Also
  TJvProgressComponent.ProgressMax, TJvProgressComponent.ProgressMin,
  TJvProgressComponent.ProgressPosition, TJvProgressComponent.ProgressStepIt

----------------------------------------------------------------------------------------------------
@@TJvProgressComponent.ProgressStepIt
Summary
  Advances \ProgressPosition by the amount specified in the \ProgressStep property.
Description
  Call the ProgressStepIt method to increase the value of \ProgressPosition by the value of the 
  ProgressStep property. If \ProgressStep represents the size of one logical step in the process
  tracked by the progress bar, call ProgressStepIt after each logical step is completed.
See Also
  TJvProgressComponent.ProgressMax, TJvProgressComponent.ProgressMin,
  TJvProgressComponent.ProgressPosition, TJvProgressComponent.ProgressStep

