##Package: StdCtrls
##Status: Locked (Remko Bonte)
----------------------------------------------------------------------------------------------------
@@JvToolEdit.pas
Summary
  Contains the TJvToolEdit component.
<INCLUDE JVCL.UnitText.dtx>
Author
  Fedor Koshevnikov, Igor Pavluk and Serge Korolev
----------------------------------------------------------------------------------------------------
@@DateFormatChanged
<TITLE DateFormatChanged procedure>
Summary
  Updates all combo date edit control's masks upon chaning of the date format.
Description
  Write here a description
See Also
  List here other properties, methods (comma seperated)
  Remove the 'See Also' section if there are no references
----------------------------------------------------------------------------------------------------
@@EComboEditError
Summary
  Write here a summary (1 line)
Description
  Write here a description
----------------------------------------------------------------------------------------------------
@@EditorTextMargins
<TITLE EditorTextMargins function>
Summary
  Write here a summary (1 line)
Description
  Write here a description
Parameters
  Editor - Description for this parameter
See Also
  List here other properties, methods (comma seperated)
  Remove the 'See Also' section if there are no references
----------------------------------------------------------------------------------------------------
@@PaintComboEdit
<TITLE PaintComboEdit function>
Summary
  Write here a summary (1 line)
Description
  Write here a description
Parameters
  Editor        - Description for this parameter
  AText         - Description for this parameter
  AAlignment    - Description for this parameter
  StandardPaint - Description for this parameter
  ACanvas       - Description for this parameter
  Msg           - Description for this parameter
See Also
  List here other properties, methods (comma seperated)
  Remove the 'See Also' section if there are no references
----------------------------------------------------------------------------------------------------
@@PaintEdit
<TITLE PaintEdit function>
Summary
  Write here a summary (1 line)
Description
  Write here a description
Parameters
  Editor            - Description for this parameter
  AText             - Description for this parameter
  AAlignment        - Description for this parameter
  PopupVisible      - Description for this parameter
  ButtonWidth       - Description for this parameter
  DisabledTextColor - Description for this parameter
  StandardPaint     - Description for this parameter
  ACanvas           - Description for this parameter
  Msg               - Description for this parameter
See Also
  List here other properties, methods (comma seperated)
  Remove the 'See Also' section if there are no references
----------------------------------------------------------------------------------------------------
@@TCalendarStyle
<TITLE TCalendarStyle type>
Summary
  Enumerates calendar styles.
Description
  Use the TCalendarStyle to specify a specific calendar style.
See Also
  TJvCustomDateEdit.CalendarStyle
@@TCalendarStyle.csPopup
  The calendar is a pop-up window.
@@TCalendarStyle.csDialog
  The calendar is a modal dialog.
----------------------------------------------------------------------------------------------------
@@TDirDialogKind
<TITLE TDirDialogKind type>
Summary
  Enumerates directory dialog kinds.
Description
  Use the TDirDialogKind type to specify a specific directory dialog kind.
See Also
  TJvDirectoryEdit.DialogKind
@@TDirDialogKind.dkVCL
  VCL Select Directory dialog box. 
@@TDirDialogKind.dkWin32
  Windows directory browser. 
----------------------------------------------------------------------------------------------------
@@TExecDateDialog
<TITLE TExecDateDialog type>
<COMBINE TJvCustomDateEdit.OnAcceptDate>
----------------------------------------------------------------------------------------------------
@@TExecOpenDialogEvent
<TITLE TExecOpenDialogEvent type>
Summary
  The type for event handlers on adapter action objects that are called when the action executes.
Description
  TActionExecuteEvent is the type for event handlers of an adapter action that can perform some or all of the action�s tasks when the action executes.
Parameters
  Sender - The object that represents the combo edit control.
  Name   - Description for this parameter
  Action - Description for this parameter
See Also
  TJvFileDirEdit.OnAfterDialog, TJvFileDirEdit.OnBeforeDialog
----------------------------------------------------------------------------------------------------
@@TFileDialogKind
<TITLE TFileDialogKind type>
Summary
  Enumerates file dialog kinds.
Description
  Use the TFileDialogKind type to specify a specific file dialog kind.
See Also
  TJvFilenameEdit.DialogKind, TJvFilenameEdit.Dialog
@@TFileDialogKind.dkOpen
  File-selection dialog.
@@TFileDialogKind.dkSave
  "Save As" dialog for saving files.
@@TFileDialogKind.dkOpenPicture
  Graphics-file selection dialog.
@@TFileDialogKind.dkSavePicture
  "Save As" dialog for saving graphics files.
----------------------------------------------------------------------------------------------------
@@TFileExt
<TITLE TFileExt type>
Summary
  A string that represents a file extension.
Description
  Use TFileExt to represent strings that are only used for file extensions.
----------------------------------------------------------------------------------------------------
@@TGlyphKind
<TITLE TGlyphKind type>
Summary
  Enumerates glyph kinds for combo edit controls.
Description
  Use the TGlyphKind type to specify the glyph kind for the button of a combo edit control.
See Also
  TJvCustomComboEdit.GlyphKind
@@TGlyphKind.gkCustom
  A custom glyph.
@@TGlyphKind.gkDefault
  The default glyph. When specifying gkDefault the actual glyph shown depends on the combo edit
  control. For example, a TJvDateEdit will display a calendar glyph.
@@TGlyphKind.gkDropDown
  A dropdown button.
@@TGlyphKind.gkEllipsis
  Ellipsis.
----------------------------------------------------------------------------------------------------
@@TJvComboEdit
<TITLEIMG TJvComboEdit>
JVCLInfo
  GROUP=JVCL.EditsMemosAndCombos.Edits,JVCL.Buttons.Push
  FLAG=Component
Summary
  Edit control with a button.
Description
  Write here a description
----------------------------------------------------------------------------------------------------
@@TJvCustomComboEdit
Summary
  Base class for edit controls with a button.
Description
  TJvCustomComboEdit encapsulates the behavior common to all components for editing text by introducing 
  methods and properties that provide:

  * Basic text editing functions such as selecting text, modifying selected text, and case conversions.
  * Ability to respond to changes in the contents of the text.
  * Access control of the text for making it read-only or introducing a password character to hide 
    the actual value.

  Do not create instances of TJvCustomComboEdit. Use TJvCustomComboEdit as a base class when declaring
  control objects that permit the user to enter or modify text. Properties and methods of TJvCustomComboEdit 
  provide basic behavior that descendant classes inherit as well as behavior that components can 
  override to customize their behavior.
----------------------------------------------------------------------------------------------------
@@TJvCustomComboEdit.Alignment
Summary
  Determines how the text in the edit component is aligned.
Description
  Use Alignment to indicate text alignment in the edit box. These are the possible values:

  <TABLE>
  Value           Meaning
  -----           -------
  taLeftJustify   Align text on the left side in the edit control.
  taCenter        Center the text in the edit control.
  taRightJustify  Align text on the right side in the edit control.
  </TABLE>
----------------------------------------------------------------------------------------------------
@@TJvCustomComboEdit.AlwaysEnable
Summary
  Specifies whether the button is always enabled.
Description
  If you set AlwaysEnable to true, an user can always click the button of the combo edit control, 
  even if ReadOnly is set to true.
See Also
  ReadOnly
----------------------------------------------------------------------------------------------------
@@TJvCustomComboEdit.Button
Summary
  Gives access to the button object of the combo edit control.
Description
  Use Button to access the combo edit control's button. Some properties of the button are accessable
  via the combo edit control's properties, such as ButtonFlat, ButtonWidth etc. But you may want
  to access property Button to get full access to the button object.
See Also
  ButtonFlat, ButtonHint, ButtonWidth
----------------------------------------------------------------------------------------------------
@@TJvCustomComboEdit.ButtonFlat
Summary
  Determines whether the button has a 3D border that provides a raised or lowered look.
Description
  Set Flat to true to remove the raised border when the button is unselected and the lowered border
  when the button is clicked or selected. When Flat is true, use separate bitmaps for the different
  button states to provide visual feedback to the user about the button state.
See Also
  Button, ButtonFlat, ButtonHint, ButtonWidth
----------------------------------------------------------------------------------------------------
@@TJvCustomComboEdit.ButtonHint
Summary
  Contains the text string that can appear when the user moves the mouse over the button.
Description
  Use the ButtonHint property to provide a string of help text as a Help Hint for the combo edit
  control's button.
See Also
  Button, ButtonFlat, ButtonWidth
----------------------------------------------------------------------------------------------------
@@TJvCustomComboEdit.ButtonWidth
Summary
  Specifies the width of the button.
Description
  Use ButtonWidth to specify the width of the combo edit control's button.
See Also
  Button, ButtonFlat, ButtonHint
----------------------------------------------------------------------------------------------------
@@TJvCustomComboEdit.ClickKey
Summary
  Specifies the key combination users can type to quickly access the menu item.
Description
  Use ClickKey to allow the user to type a key combination instead of clicking the control's button. 

  When setting shortcuts at design time, the Object Inspector provides a long list of key
  combinations to choose from. To create shortcuts at runtime, use the global ShortCut
  or TextToShortCut function. For example, the following line creates a shortcut, Ctrl+C
  and assigns it to a combo edit control named JvComboEdit1.

  <AUTOLINK OFF>
  <CODE>
  JvComboEdit1.ClickKey := ShortCut(Word('C'), [ssCtrl]);
  </CODE>
  <AUTOLINK ON>
----------------------------------------------------------------------------------------------------
@@TJvCustomComboEdit.ClipboardCommands
Summary
  Specifies the clipboard commands that are enabled for the control.
Description
  Use ClipboardCommands to specify which clipboard commands are allowed for the control.
  ClipboardCommands can't be set to []. If ClipboardCommands is set to [caCopy] then property ReadOnly 
  is set to true.
----------------------------------------------------------------------------------------------------
@@TJvCustomComboEdit.DirectInput
Summary
  Specifies whether the user can use the edit box of the combo edit.
Description
  Use DirectInput to read or change whether the user can use the edit box of the combo edit. By setting 
  DirectInput to false, the edit box of the combo edit will become read-only, and the user can only
  use the button of the combo edit.
See Also
  AlwaysEnable, ReadOnly
----------------------------------------------------------------------------------------------------
@@TJvCustomComboEdit.DisabledColor
Summary
  Specifies the background color of the control displayed when the control isn't enabled.
Description
  Use DisabledColor to read or change the background color displayed when the control isn't enabled.

  Set Enabled to false to disable the control.
See Also
  DisabledTextColor
----------------------------------------------------------------------------------------------------
@@TJvCustomComboEdit.DisabledTextColor
Summary
  Specifies the color of the text when the control isn't enabled.
Description
  Use DisabledTextColor to specify the color of the text characters (as opposed to the background color),
  when the control isn't enabled.

  Set Enabled to false to disable the control.
See Also
  DisabledColor
----------------------------------------------------------------------------------------------------
@@TJvCustomComboEdit.DoClick
Summary
  Simulates a mouse click, as if the user had clicked the button.
Description
  Call DoClick to simulate a mouse click on the button. If the combo edit control has an OnButtonClick
  event handler, DoClick allows an application to simulate a click on the button.
See Also
  OnButtonClick
----------------------------------------------------------------------------------------------------
@@TJvCustomComboEdit.Glyph
Summary
  Specifies the bitmap that appears on the button of the combo edit control.
Description
  Use the Open dialog box that appears as an editor in the Object Inspector to choose a bitmap file
  (with a .BMP extension) to use on the button, or specify a TBitmap object at runtime.

  You can provide up to four images within a single bitmap. All images must be the same size and next
  to each other in a row. Edit buttons display one of these images depending on their state.
See Also
  GlyphKind, NumGlyphs
----------------------------------------------------------------------------------------------------
@@TJvCustomComboEdit.GlyphKind
Summary
  Specifies the kind of glyph for the edit button.
Description
  Use GlyphKind to specify the appearance of the button of the edit button control.
See Also
  Glyph, NumGlyphs
----------------------------------------------------------------------------------------------------
@@TJvCustomComboEdit.NumGlyphs
Summary
  Indicates the number of images that are in the graphic specified in the Glyph property.
Description
  If you have multiple images in a bitmap, you must specify the number of images that are in the
  bitmap with the NumGlyphs property. The default value is 1.
See Also
  Glyph, GlyphKind
----------------------------------------------------------------------------------------------------
@@TJvCustomComboEdit.OnButtonClick
Summary
  Occurs when the user clicks the button.
Description
  Write an OnButtonClick handler to respond to the user clicking the button of the combo edit control.
See Also
  OnKeyDown
----------------------------------------------------------------------------------------------------
@@TJvCustomComboEdit.OnKeyDown
Summary
  Occurs when a user presses any key while the control has focus. 
Description
  Use the OnKeyDown event handler to specify special processing to occur when a key is pressed. The
  OnKeyDown handler can respond to all keyboard keys, including function keys and keys combined with
  the Shift, Alt, and Ctrl keys, and pressed mouse buttons.
See Also
  OnButtonClick
----------------------------------------------------------------------------------------------------
@@TJvCustomComboEdit.PopupAlign
Summary
  Specifies how the pop-up window is aligned relative to its edit box.
Description
  Use PopupAlign to specify the position of the pop-up window relative to the edit region of the
  lookup combo box. The pop-up window can be aligned to the left or right, or centered with the edit box.
----------------------------------------------------------------------------------------------------
@@TJvCustomComboEdit.PopupVisible
Summary
  Indicates whether the pop-up window is visible.
Description
  Read PopupVisible to determine whether the pop-up window is visible.
  
  The pop-up window is normally shown when the user clicks the button of the combo edit control.
----------------------------------------------------------------------------------------------------
@@TJvCustomComboEdit.ReadOnly
Summary
  Determines whether the user can change the text of the combo edit control.
Description
  To restrict the edit control to display only, set the ReadOnly property to true. Set ReadOnly to
  false to allow the contents of the edit control to be edited.

  Setting ReadOnly to true ensures that the text is not altered, while still allowing the user to
  select text. The selected text can then be manipulated by the application, or copied to the Clipboard.
See Also
  AlwaysEnable, DirectInput
----------------------------------------------------------------------------------------------------
@@TJvCustomDateEdit
Summary
  Base class for the TJvDateEdit control.
Description
  Use TCommonCalendar as a base class when creating custom controls that represent dates in a
  calendar-like format. Do not create instances of TCommonCalendar. Instead, to put a calendar
  control on a form, use a TCommonCalendar descendant such as TDateTimePicker or TMonthCalendar.
----------------------------------------------------------------------------------------------------
@@TJvCustomDateEdit.BlanksChar
Summary
  Specifies the character used to represent unentered characters in the date mask.
Description
  Use BlanksChar to specify the character used to represent unentered characters in the date mask.

  The default value of BlanksChar is the space.
See Also
  <LINK GetDateMask>, YearDigits
----------------------------------------------------------------------------------------------------
@@TJvCustomDateEdit.CalendarHints
Summary
  Provides a way to customize the Help Hints for the buttons on the pop-up calendar window.
Description
  Use the CalendarHints property to supply Help Hints of your choosing for the individual
  buttons.
  Each button has a default Help Hint. CalendarHints allow the values of any or all of these default
  Help Hints to be replaced by customized hints. 

  CalendarHints is a string list. Each hint is a string. The first string in the string list becomes the
  Help Hint for the first button on the navigator (the Previous Year button). The fourth hint becomes the
  Help Hint for the fourth button (the Next Year button). 

  When specifying CalendarHints at runtime, enter an empty string for any Help Hint that should keep the
  default value. Simply leave the line blank when using the string list property editor of the
  Object Inspector for the CalendarHints property.
----------------------------------------------------------------------------------------------------
@@TJvCustomDateEdit.CalendarStyle
Summary
  Specifies the style of the pop-up calendar.
Description
  Use CalendarStyle to specify the style of the pop-up calendar, that is displayed when the user
  clicks the button of the combo edit control. The calendar can either be a pop-up window or a 
  modal dialog.
See Also
  TJvCustomComboEdit.PopupAlign, DialogTitle, PopupColor
----------------------------------------------------------------------------------------------------
@@TJvCustomDateEdit.CheckOnExit
Summary
  Specifies whether to check the control's text when the user attempts to leave the control.
Description
  Use CheckOnExit to specify whether to validate the control's text when the user attemps to leave
  the combo control. If set to true, the control calls CheckValidateDate on exit of the control.
See Also
  CheckValidDate, Date, DateAutoBetween
----------------------------------------------------------------------------------------------------
@@TJvCustomDateEdit.CheckValidDate
Summary
  Validates the text of the combo control against the current date mask.
Description
  CheckValidDate validates the combo control's text. If the text does not match the specifications
  of the date mask, CheckValidDate raises an exception.
  
  Call <LINK GetDateMask> to retrieve the date mask. Use CheckOnExit to specify whether to check the text
  on exit of the control.
See Also
  CheckOnExit, Date, DateAutoBetween
----------------------------------------------------------------------------------------------------
@@TJvCustomDateEdit.Date
Summary
  Indicates the file name displayed in the edit portion of the combo edit control.
  Indicates the date that is marked on the calendar.
Description
  Use Date to get or set the date that is marked on the calendar. The value of Date must lie within
  the range specified by the MaxDate and MinDate properties.

  If MultiSelect is true, the selected range of dates goes from Date to EndDate.
See Also
  CheckValidDate, DateAutoBetween
----------------------------------------------------------------------------------------------------
@@TJvCustomDateEdit.DateAutoBetween
Summary
  Specifies whether to automatically alter the entered date to ensure it is between MinDate and MaxDate.
Description
  Set DateAutoBetween to true to let the control alter the entered date to ensure it is between
  MinDate and MaxDate. Set DateAutoBetween to false to let the control raise an exception when the
  entered date is not between MinDate and MaxDate.
Note
  If you specify the value 0 for MinDate, then the control won't check whether the entered date is
  equal or after the MinDate. Same for MaxDate.
See Also
  Date
----------------------------------------------------------------------------------------------------
@@TJvCustomDateEdit.DefaultToday
Summary
  Specifies whether to use todays day as default.
Description
  Set DefaultToday to true to use todays date as default. For example, if the Text of the control
  is empty then Date will return todays date. Set DefaultToday to false to use the null value
  (12/30/1899 12:00 am) as default.
See Also
  Date
----------------------------------------------------------------------------------------------------
@@TJvCustomDateEdit.DialogTitle
Summary
  Specifies the text in the dialog�s title bar.
Description
  Use Title to specify the text that appears in the dialog�s title bar.
Note 
  Only used if CalendarStyle is csDialog.
See Also
  TJvCustomComboEdit.PopupAlign, CalendarStyle, PopupColor
----------------------------------------------------------------------------------------------------
@@TJvCustomDateEdit.Formatting
Summary
  Indicates whether the control is setting it's text.
Description
  The control checks Formatting internally to determine whether it is busy formatting it's text.
----------------------------------------------------------------------------------------------------
@@TJvCustomDateEdit.GetDateMask
Summary
  Returns the mask that represents what text is valid for the control.
Description
  Call GetDateMask to retrieve the mask that represents what text is valid for the control.
  
  The value of the mask depends on the values of BlanksChar and YearDigits, and on 
  the date format specified in the Windows control panel.
Return value
  Returns the mask that represents what text is valid for the control.
See Also
  BlanksChar, YearDigits
----------------------------------------------------------------------------------------------------
@@TJvCustomDateEdit.MaxDate
Summary
  Indicates the maximum date to which users can scroll the calendar.
Description
  Use MaxDate to get or set the maximum date to which users can scroll the calendar. The values of
  the Date and EndDate properties cannot exceed MaxDate.
See Also
  Date, MaxDate
----------------------------------------------------------------------------------------------------
@@TJvCustomDateEdit.MinDate
Summary
  Indicates the minimum date that can be selected.
Description
  Use MinDate to get or set the minimum date that can be selected.
See Also
  Date, MaxDate
----------------------------------------------------------------------------------------------------
@@TJvCustomDateEdit.OnAcceptDate
Summary
  Occurs after the user exits a modal dialog by selecting �OK� or when a modeless dialog is 
  successfully displayed.
  Occurs when the user types directly into the component�s edit box.
Description
  Write an OnAccept event handler to make use the dialog results after it executes. When the action fires,
  it executes its associated dialog. OnAccept occurs when the dialog component�s Execute method returns
  true. Thus, if the dialog is modal, OnAccept occurs when the user closes the dialog by clicking the
  OK button. If the dialog is modeless, OnAccept occurs when the dialog first appears.

  For modal dialogs, use the OnAccept event to read the user�s selections from the dialog and act on them.
  TCommonDialogAction descendants introduce a Dialog property that the event handler can use to obtain
  information about the user�s selections.
  --
  This event occurs only if ParseInput is set to true. Write an OnUserInput event handler implement
  special processing that needs to occur when the user types directly into the TDateTimePicker (rather
  than selecting with the drop-down calendar or scroll arrows). UserString is the string that the user
  types, and DateAndTime represents the value of the Date or Time property.
Parameters
  Sender - Description for this parameter
  ADate  - Description for this parameter
  Action - Description for this parameter
See Also
  List here other properties, methods (comma seperated)
  Remove the 'See Also' section if there are no references
----------------------------------------------------------------------------------------------------
@@TJvCustomDateEdit.PopupColor
Summary
  Specifies the color of the pop-up calendar.
Description
  Use PopupColor to specify the color of the pop-up calendar.
Note
  Only used if CalendarStyle is csPopup.
See Also
  TJvCustomComboEdit.PopupAlign, DialogTitle, CalendarStyle
----------------------------------------------------------------------------------------------------
@@TJvCustomDateEdit.StartOfWeek
Summary
  Specifies the day that is the left-most day in the pop-up calendar.
Description
  Use StartOfWeek to specify the day that is the left-most day in the pop-up calendar.

  The default value of StartOfWeek is monday.
See Also
  Weekends
----------------------------------------------------------------------------------------------------
@@TJvCustomDateEdit.UpdateMask
Summary
  Write here a summary (1 line)
Description
  Write here a description
See Also
  List here other properties, methods (comma seperated)
  Remove the 'See Also' section if there are no references
----------------------------------------------------------------------------------------------------
@@TJvCustomDateEdit.ValidateEdit
Summary
  Validates the text of the combo control against the current date mask.
Description
  ValidateEdit validates the combo control's text. If the text does not match the specifications
  of the date mask, ValidateEdit raises an exception.
  
  Call <LINK GetDateMask> to retrieve the date mask. Use CheckOnExit to specify whether to check the text
  on exit of the control.
See Also
  GetDateMask, CheckOnExit
----------------------------------------------------------------------------------------------------
@@TJvCustomDateEdit.WeekendColor
Summary
  The text color of weekend days in the calendar.
Description
  WeekendColor determines the text color of weekend days in the calendar.
  
  Use Weekends to specify the days that are marked as weekend days.
See Also
  Weekends
----------------------------------------------------------------------------------------------------
@@TJvCustomDateEdit.Weekends
Summary
  Specifies the days marked as weekend days in the pop-up calendar.
Description
  Use Weekends to specify the days marked as weekend days in the pop-up calendar.
  
  Weekend days are displayed in the color specified by WeekendColor.
  The default value of Weekends is [sunday].
See Also
  StartOfWeek, WeekendColor
----------------------------------------------------------------------------------------------------
@@TJvCustomDateEdit.YearDigits
Summary
  Specifies the number of digits used for the year.
Description
  Use YearDigits to specify the number of digits used for the year part of the date displayed in 
  combo edit control.
See Also
  BlanksChar, YearDigits
----------------------------------------------------------------------------------------------------
@@TJvDateEdit
<TITLEIMG TJvDateEdit>
JVCLInfo
  GROUP=JVCL.DateTime.EditsCombos,JVCL.EditsMemosAndCombos.Edits,JVCL.Buttons.Push
  FLAG=Component
Summary
  Edit control with a button that displays a calendar.
Description
  Write here a description
----------------------------------------------------------------------------------------------------
@@TJvDirectoryEdit
<TITLEIMG TJvDirectoryEdit>
JVCLInfo
  GROUP=JVCL.EditsMemosAndCombos.Edits,JVCL.Buttons.Push
  FLAG=Component
Summary
  Edit control with a button that displays a directory selection dialog.
Description
  Write here a description
See Also
  TJvFilenameEdit
----------------------------------------------------------------------------------------------------
@@TJvDirectoryEdit.DialogKind
Summary
  Specifies the kind of the dialog associated with the combo edit control.
Description
  Use the DialogKind property to specify the kind of dialog that is displayed when the user presses
  the button of the edit control.
See Also
  DialogOptions, DialogText
----------------------------------------------------------------------------------------------------
@@TJvDirectoryEdit.DialogOptions
Summary
  Write here a summary (1 line)
Description
  The Options parameter is a set of values. If Options is the empty set, the user can only select
  a directory that already exists. No edit box is provided for the user to enter a new directory name.
  If Options is not empty, the included values determine how the dialog responds when the user types
  a nonexistent directory name.
Note
  Only used if Dialogkind is dkVCL.
See Also
  DialogKind, DialogText
----------------------------------------------------------------------------------------------------
@@TJvDirectoryEdit.DialogText
Summary
  Specifies the text in the dialog�s title bar.
Description
  Use Title to specify the text that appears in the dialog�s title bar.
Note
  Only used if DialogKind is dkWin32.
See Also
  DialogKind, DialogOptions
----------------------------------------------------------------------------------------------------
@@TJvDirectoryEdit.InitialDir
Summary
  Determines the selected directory when the dialog opens.
Description
  To make an item selected by default in the dialog�s box, assign a value to InitialDir in the Object Inspector
  or in program code. 
  
  If the user selects a directory in the dialog and clicks OK, then InitialDir directory is set to that directory.
Note
  Only used if the edit text is empty when the dialog is shown. If the edit text is <i>not</i> empty
  then that text determines the to be selected directory.
----------------------------------------------------------------------------------------------------
@@TJvDirectoryEdit.MultipleDirs
Summary
  Allows users to select more than one file in the dialog.

Description
  Write here a description
See Also
  List here other properties, methods (comma seperated)
  Remove the 'See Also' section if there are no references
----------------------------------------------------------------------------------------------------
@@TJvFileDirEdit
Summary
  Base class for the controls TJvDirectoryEdit and TJvFilenameEdit.
Description
  Use TJvFileDirEdit as a base class when creating custom controls that represent dates in a
  calendar-like format. Do not create instances of TJvFileDirEdit. Instead, to put a calendar
  control on a form, use a TJvFileDirEdit descendant such as TJvDirectoryEdit or TJvFilenameEdit.
See Also
  TJvFilenameEdit, TJvDirectoryEdit
----------------------------------------------------------------------------------------------------
@@TJvFileDirEdit.AcceptFiles
Summary
  Specifies whether the combo edit control accepts dropped files.
Description
  Set AcceptFiles to true to accept dropped files or set it to false to discontinue accepting
  dropped files.
See Also
  TJvDirectoryEdit.MultipleDirs, OnDropFiles
----------------------------------------------------------------------------------------------------
@@TJvFileDirEdit.LongName
Summary
  Indicates the long filename of the file or directory displayed in the combo edit control.
Description
  Use LongName to retrieve the name of the file or directory displayed in the edit box of
  the combo edit control, in the long file name format.
Note
  The file entered in the combo edit control <i>must</i> exist, otherwise an empty string is returned.
See Also
  ShortName, TJvFilenameEdit.FileName
----------------------------------------------------------------------------------------------------
@@TJvFileDirEdit.OnAfterDialog
Summary
  Occurs after an adapter action has executed.
Description
  Implement OnAfterExecuteAction to do something after the action has executed. The Params parameter
  contains the adapter action parameters (name/value pairs) that were passed in the HTTP request.
See Also
  OnBeforeDialog
----------------------------------------------------------------------------------------------------
@@TJvFileDirEdit.OnBeforeDialog
Summary
  Called before the adapter action is executed.
Description
  Implement OnBeforeExecute to do something before the action is executed. The Params parameter
  contains the adapter action parameters (name/value pairs) that were passed in the HTTP request.
  Set the Handled flag to prevent the action from being executed. OnBeforeExecute is called by
  DoBeforeExecuteActionRequest.
See Also
  OnAfterDialog
----------------------------------------------------------------------------------------------------
@@TJvFileDirEdit.OnDropFiles
Summary
  Occurs when the user drops files on the combo edit control.
Description
  Write an OnDropFiles event handler to respond when the user drops files on the control.
Note
  Set AcceptFiles to true, otherwise no OnDrop event is fired.
See Also
  AcceptFiles
----------------------------------------------------------------------------------------------------
@@TJvFileDirEdit.ShortName
Summary
  Indicates the short filename of the file or directory displayed in the combo edit control.
Description
  Use ShortName to retrieve the name of the file or directory displayed in the edit box of
  the combo edit control, in the 8.3 (filename.ext) file name format. 
Note
  The file entered in the combo edit control <i>must</i> exist, otherwise an empty string is returned.
See Also
  LongName, TJvFilenameEdit.FileName
----------------------------------------------------------------------------------------------------
@@TJvFilenameEdit
<TITLEIMG TJvFilenameEdit>
JVCLInfo
  GROUP=JVCL.EditsMemosAndCombos.Edits,JVCL.Buttons.Push
  FLAG=Component
Summary
  Edit control with a button that displays a file selection dialog.
Description
  Write here a description
See Also
  TJvDirectoryEdit
----------------------------------------------------------------------------------------------------
@@TJvFilenameEdit.DefaultExt
Summary
  Specifies a default file extension.
Description
  DefaultExt specifies a file extension that is appended automatically to the selected file name,
  unless the selected file name already includes a registered extension. If the user selects a file
  name with an extension that is unregistered, DefaultExt is appended to the unregistered extension.

  Extensions longer than three characters are not supported. Do not include the period (.) that
  divides the file name and its extension.
See Also
  Dialog
----------------------------------------------------------------------------------------------------
@@TJvFilenameEdit.Dialog
Summary
  Gives access to the combo edit control's dialog.
Description
  Use Dialog to access the combo edit control's dialog, but most properties of the dialog are accessable
  via properties of the combo edit control, such as DefaultExt, DialogOptions etc.
  
  The class type of the combo edit control's dialog depends on the value of DialogKind.
See Also
  DialogKind
----------------------------------------------------------------------------------------------------
@@TJvFilenameEdit.DialogFiles
Summary
  Returns a list of the selected file names.
Description
  FileName is a string list that contains each selected file name with its full directory path. 
  (To let users select multiple file names, set the ofAllowMultiSelect flag in Options.) Use 
  properties and methods for string lists to traverse this list of files and read individual items.

  The example below assigns the list of files in Files to the Items property of a TListBox component.

  <AUTOLINK OFF>
  <CODE>
  ListBox1.Items.Assign(JvFilenameEdit1.DialogFiles);
  </CODE>
  </AUTOLINK ON>
See Also
  Dialog, DialogOptions
----------------------------------------------------------------------------------------------------
@@TJvFilenameEdit.DialogKind
Summary
  Specifies the kind of the dialog associated with the combo edit control.
Description
  Use the DialogKind property to specify the kind of dialog that is displayed when the user presses
  the button of the edit control.
See Also
  Dialog, DialogOptions, DialogTitle
----------------------------------------------------------------------------------------------------
@@TJvFilenameEdit.DialogOptions
Summary
  Determines the appearance and behavior of the file-selection dialog.
Description
  Use the DialogOptions property to customize the appearance and functionality of the dialog
  displayed when the user presses the button of the edit control.
See Also
  Dialog, DialogKind, DialogTitle
----------------------------------------------------------------------------------------------------
@@TJvFilenameEdit.DialogTitle
Summary
  Specifies the text in the dialog�s title bar.
Description
  Use Title to specify the text that appears in the dialog�s title bar.
See Also
  Dialog, DialogKind, DialogOptions
----------------------------------------------------------------------------------------------------
@@TJvFilenameEdit.FileEditStyle
Summary
  Determines the style of the file-selection dialog. (Obsolete.)
Description
  FileEditStyle is maintained for compatibility with older versions of the VCL. It has no effect.
----------------------------------------------------------------------------------------------------
@@TJvFilenameEdit.FileName
Summary
  Indicates the file name displayed in the edit portion of the combo edit control.
Description
  Set FileName to initialize the combo edit to a particular file name. Read FileName to obtain the 
  name of the file that was selected by the user.

  When the user uses the File-selection dialog to select a new file, the selected file becomes the 
  value of the FileName property. The value of the Text property also changes to the new file when 
  the FileName property value changes.
See Also
  TJvFileDirEdit.LongName, TJvFileDirEdit.ShortName
----------------------------------------------------------------------------------------------------
@@TJvFilenameEdit.Filter
Summary
  Determines the file masks (filters) available in the dialog.
Description
  See Delphi help at TOpenDialog.Filter for more info.
See Also
  FilterIndex
----------------------------------------------------------------------------------------------------
@@TJvFilenameEdit.FilterIndex
Summary
  Determines which filter is selected by default when the dialog opens.
Description
  FilterIndex determines which of the file types in Filter is selected by default when the dialog
  opens. Set FilterIndex to 1 to choose the first file type in the list as the default, or set
  FilterIndex to 2 to choose the second file type as the default, and so forth. If the value of
  FilterIndex is out or range, the first file type listed in Filter is the default.
See Also
  Filter
----------------------------------------------------------------------------------------------------
@@TJvFilenameEdit.HistoryList
Summary
  Maintains a list of previously selected files. (Obsolete.)
Description
  HistoryList is maintained for compatibility with older versions of TOpenDialog. It is not used.
----------------------------------------------------------------------------------------------------
@@TJvFilenameEdit.InitialDir
Summary
  Determines the current directory when the dialog opens.
Description
  See Delphi help at TOpenDialog.InitialDir for more info.
----------------------------------------------------------------------------------------------------
@@TPopupAlign
<TITLE TPopupAlign type>
Summary
  Represents the position of a popup menu relative to the control.
Description
  The TPopupAlign type represents the position of a popup menu relative to the control.
See Also
  TJvCustomComboEdit.PopupAlign
@@TPopupAlign.epaRight
  The pop-up menu appears with its top right corner under the mouse.
@@TPopupAlign.epaLeft
  The pop-up menu appears with its top left corner under the mouse.
----------------------------------------------------------------------------------------------------
@@TYearDigits
<TITLE TYearDigits type>
Summary
  Enumerates ways to display the year part of a date.
Description
  Use the TYearDigits type to specify the number of digits to be used to display the year part of a date.
See Also
  TJvCustomDateEdit.YearDigits
@@TYearDigits.dyDefault
  Uses either 2 or 4 digits for the year, depending on the number of digits used to display the year
  in the short date format specified in the Windows control panel.
@@TYearDigits.dyFour
  Uses 4 digits for the year, ie 1999, 2000, 2001 etc.
@@TYearDigits.dyTwo
  Uses 2 digits for the year, ie 99, 00, 01 etc.
