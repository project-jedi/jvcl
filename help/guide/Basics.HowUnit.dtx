@@JVCL.HWG.Basics.HowUnit
<GROUP JVCL.HWG.Basics>
<TITLE Describing units>
<TOPICORDER 200>
  Units are always documented in very specific, constant way.
  
  * Unit topic *
  Describing the unit starts with the unit file name (including the extension) as a topic ID. Note
  that Doc-O-Matic is case sensitive, so use the same casing as the Explorer shows you.
<CODE>
\@@JvSearchFiles.pas
</CODE>

  * Placing it in the file reference node *
  Following the topic ID we tell Doc-O-Matic to put this topic in the file reference node of the
  contents. This is done by adding the <B>GROUP</B> tag as specified in the example. This line will
  always be the same.
<CODE>
\<GROUP JVCL.FileRef>
</CODE>
  
  * Summary *
  Next is the <B>summary</B> section. The summary section will usually be in one of the following forms:
    
    * Contains all code for TJvSearchFiles.
    * Contains the TJvSearchFiles component.
<CODE>
\Summary
  Contains the TJvSearchFiles component.
</CODE>
 
  * Description *
  Because all unit documentation have the same description, regardless of the unit and its contents,
  this standard text has been placed in a file. Using the <B>INCLUDE</B> tag we pull that
  description in. Again, this line will always be the same.
<CODE>
\<INCLUDE JVCL.UnitText.dtx>
</CODE>

  * Original author *
  The last thing to do is add the Author section. If you can not deduct the author from the source
  file, specify <B>Unknown</B> as author. 
<CODE>
\Author
  Peter Th�rnqvist
</CODE>
  If multiple authors are mentioned as original authors, add them as a comma separated list:
<CODE>
\Author
  Author1, Author2, Author3
</CODE>
  Note that there are spaces between the comma and the author name.
----------------------------------------------------------------------------------------------------
