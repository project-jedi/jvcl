##Package: DB
##Status: 
----------------------------------------------------------------------------------------------------
@@JvDBLookup.pas
Summary
    Contains the TJvDBLookupCombo, TJvDBLookupEdit and TJvDBLookupList controls.
Author
  Fedor Koshevnikov
  Igor Pavluk
  Serge Korolev
----------------------------------------------------------------------------------------------------
@@TJvLookupControl.DisplayValue
Summary
  Indicates the displayed value.
Description
  Inspect DisplayValue to determine the value that appears in the edit region of the lookup control.
  The user can change the value of DisplayValue by selecting a new value if ReadOnly is false and the
  dataset is in edit mode.
  
  DisplayValue corresponds to the value of the lookup field on the current record. If this value is
  empty than DisplayValue is set to the DisplayEmpty string.
See Also
  TJvLookupControl.DisplayEmpty

----------------------------------------------------------------------------------------------------
@@TJvLookupControl.DisplayEmpty
Summary
  Specifies a text to display when the lookup value is empty.
Description
  Use DisplayEmpty to display a string when an empty lookup value is selected. Property EmptyValue
  specifies which value is the empty value.
  
  Also when the drop-down list of the control is shown, an extra fixed row is displayed with the
  string
  DisplayEmpty as text. When the user selects this row the lookup value is set to EmptyValue (or to
  Null if EmptyValue is the empty string and EmptyStrIsNull is True).
  
  Set DisplayEmpty to an empty string (the default) to disable this feature.
See Also
  TJvLookupControl.EmptyItemColor, TJvLookupControl.EmptyValue

----------------------------------------------------------------------------------------------------
@@TJvLookupControl.DataSource
Summary
  Links the lookup control to the dataset that contains the DataField.
Description
  Use DataSource to specify the data source component that identifies the dataset the lookup control
  represents. The lookup control represents a field in one table by displaying the value of a
  corresponding field in another table. DataSource links to the dataset for the field the control
  represents, as opposed to the LookupSource, which links the lookup control to the lookup table that
  holds the field that is actually displayed. If the lookup control is used to edit data, the field
  that actually gets changed is the one in the DataSource.
See Also
  TJvLookupControl.DataField, TJvLookupControl.LookupSource

----------------------------------------------------------------------------------------------------
@@TJvLookupControl.ClearValue
Summary
  Clears the field value.
Description
  Use ClearValue to clear the field value using the EmptyValue.

----------------------------------------------------------------------------------------------------
@@TJvLookupControl.DataField
Summary
  Identifies the field whose value is represented by the lookup control.
Description
  Use DataField to bind the lookup control to a field in the dataset specified by the DataSource
  property. The DataField is the field whose value can be set by the lookup control, not the field
  which supplies the lookup values that are displayed by the lookup control.
  
  If DataField specifies a lookup field component, there is no need to set the LookupField or
  LookupField properties, as the field component contains all the information about the relationship
  between the data field value and the lookup value.
See Also
  TJvLookupControl.DataSource, TJvLookupControl.KeyValue

----------------------------------------------------------------------------------------------------
@@TJvLookupControl
Summary
  Base class for Lookup support
  
  TJvLookupControl is the base class for controls that provide the user with a list of lookup items
  for filling in fields that require data from another dataset.
Description
  Use TJvLookupControl as a base class when defining controls that permit the user to set a field
  value by selecting a corresponding value from another dataset. The TJvDBLookupCombo and
  TJvDBLookupList controls are both descended from the TJvLookupControl and provide the user with
  automatic lookup capabilities that are implemented in the TJvLookupControl object.
  
  Do not create instances of TJvLookupControl. Instead, instantiate a descendant of TJvLookupControl
  to allow the user to set field values from a set of lookup values.

----------------------------------------------------------------------------------------------------
@@TJvDBLookupList.RowCount
Summary
  Specifies how many rows are visible in the list box.
Description
  Set RowCount to the number of rows displayed in the lookup list box. The list box is resized to
  match the specified number of rows. Resizing the list box (such as by changing its Height property)
  automatically changes the value of RowCount to match the new height.

----------------------------------------------------------------------------------------------------
@@TJvDBLookupList.DrawItemText
Summary
  Write here a summary (1 line)
Description
  Write here a description
Parameters
  Canvas   - Description for this parameter
  Rect     - Description for this parameter
  Selected - Description for this parameter
  IsEmpty  - Description for this parameter
See Also
  List here other properties, methods (comma seperated)
  Remove the 'See Also' section if there are no references

----------------------------------------------------------------------------------------------------
@@TJvDBLookupList.BorderStyle
Summary
  Determines whether the lookup list box has a border.
Description
  Set BorderStyle to specify whether a border should be drawn around the lookup list box. These are
  the possible values:
  <TABLE>
  Value     Meaning
  --------  ------------------
  bsNone    No visible border
  bsSingle  Single-line border </TABLE>

----------------------------------------------------------------------------------------------------
@@TJvDBLookupList
<TITLEIMG TJvDBLookupList>
#JVCLInfo
<GROUP JVCL.ListsAndTrees.Lists,JVCL.DBAware.GridsTreesLists.Lists>
<FLAG Component>
Summary
  Provides a list of lookup items for filling in fields that require data from another dataset.
Description
  Use TJvDBLookupList to provide users with a convenient list of lookup items to set a field value
  using the values of a field in another dataset. Lookup list boxes usually display values that are a
  represent a more meaningful description of the actual field value.
  
  TJvDBLookupList provides a list of lookup items for filling in fields that require data from
  another dataset. Use TJvDBLookupList to provide users with a convenient list of lookup items to set
  a field value using the values of a field in another dataset. Lookup list boxes usually display
  values that are a represent a more meaningful description of the actual field value.
  
  The relationship between field values and the corresponding values in the lookup dataset can be set
  using the properties LookupSource, LookupField and LookupDisplay.
  
  This component provides the following:
  
  * You can select any number of fields to be displayed in the lookup list.
  * End-users can incrementally search through the lookup list by directly typing into the control.
  This is a great advantage when using lookup tables that contain hundreds of even thousands of
  records.
  * You can perform a lookup on a Query or QBE result. It even incrementally searches on the query  
  result.
  * The component does not have to be bound, or assigned, to a table's field (DataField and
  DataSource   properties) which gives you greater flexibility in using this lookup list for general
  tasks where a source table is not involved.
  
  If DataSource and DataField properties is set, when a user selects a list item, the corresponding
  field value is changed in the underlying dataset.

----------------------------------------------------------------------------------------------------
@@TJvDBLookupEdit.UseRecordCount
Summary
  Specifies whether the control is allowed to use RecordCount.
Description
  Use the UseRecordCount property to allow or forbid the control to use the RecordCount property of
  the lookup dataset. Record counting can be a costly operation, especially for SQL queries that
  return large result sets.
  
  If UseRecordCount is set to false then the position of the scrollbar's thumb of the lookup list is
  only an approximation.

----------------------------------------------------------------------------------------------------
@@TJvDBLookupEdit.PopupOnlyLocate
Summary
  Specifies when to launch a locate on the list data.
Description
  When false, typing into the Edit region does not launch a locate on the list data.
  
  If true (default) this property affects the component behavior when the edit value is changing.

----------------------------------------------------------------------------------------------------
@@TJvDBLookupEdit.OnGetImage
Summary
  OnGetImage event is triggered when an image is requested. You can also use an OnGetImage event to
  specify graphical picture to display in each item of lookup list accordingly to the contents of
  lookup source.
Description
  Write here a description

----------------------------------------------------------------------------------------------------
@@TJvDBLookupEdit.OnDropDown
Summary
  Write here a summary (1 line)
Description
  Fired when list is dropped down

----------------------------------------------------------------------------------------------------
@@TJvDBLookupEdit.OnCloseUp
Summary
  Write here a summary (1 line)
Description
  Fired when list is closed

----------------------------------------------------------------------------------------------------
@@TJvDBLookupEdit.DropDownWidth
Summary
  Specifies the width, in pixels, of the drop-down list.
Description
  Use DropDownWidth to customize the width of the drop-down list. If DropDownWidth is 0 (the
  default), the drop-down list is the same width as the combo box.
  
  DropDownWidth is useful when displaying multiple lookup fields, and therefore, multiple columns in
  the database lookup combo box.

----------------------------------------------------------------------------------------------------
@@TJvDBLookupEdit.FieldsDelimiter
Summary
  Specifies the character used to delimit the field values in the drop-down lookup list.
Description
  Use FieldsDelimiter to display another character as field delimiter than the comma character.
  
  FieldsDelimiter has only meaning if ListStyle is set to lsDelimited. Also LookupFormat must have no
  value.
See Also
  TJvDBLookupEdit.ListStyle, TJvLookupControl.LookupFormat

----------------------------------------------------------------------------------------------------
@@TJvDBLookupEdit.ListStyle
Summary
  Determines how the lookup list displays its items.
Description
  ListStyle determines how the lookup list displays its items when multiple fields specified by the
  LookupDisplay property.
  
  By default, style is lsFixed, meaning that the each display field always takes up a fixed width
  accordingly to their DisplayWidth properties. The lsDelimited style meaning that each field takes a
  variable width, and the fields are separated by the FieldsDelimiter character.
See Also
  TJvDBLookupEdit.FieldsDelimiter

----------------------------------------------------------------------------------------------------
@@TJvDBLookupEdit.LookupDisplay
Summary
  Identifies the field or fields whose values are displayed in the lookup list.
Description
  The lookup control represents a field in one table by displaying the value of one or more
  corresponding fields in another table. LookupField is the name of the field or fields in the lookup
  table that are actually displayed, as opposed to the DataField, which is the field in the
  DataSource that the lookup control actually represents, or the LookupField, which is the field in
  the lookup table with the same value as the DataField.
  
  To fully specify the list fields, both a dataset for the lookup table and the fields within that
  dataset must be defined. The LookupSource property of the lookup control specifies the dataset for
  the lookup table.
  
  LookupField can represent more than one field. Separate multiple field names with semicolons.
  
  Before specifying LookupField, specify the link between the two datasets using the LookupField
  property. If LookupField is not set, lookup controls display LookupField field values by default.
  If the DataField field is a lookup field, don't specify anything for LookupField; the data controls
  automatically use the lookup field's LookupResultField property as LookupField.

----------------------------------------------------------------------------------------------------
@@TJvDBLookupEdit.LookupDisplayIndex
Summary
  Specifies which field from the LookupField property is used for incremental searching.
Description
  When the LookupField property specifies more than one field, use LookupDisplayIndex to specify
  which of those fields is the one to use for incremental searches. For the TDBLookupComboBox object,
  LookupDisplayIndex also determines which field appears in the edit region of the combo box.
  
  LookupDisplayIndex allows the most important value to appear in a position other than the first
  when all the values of the LookupField fields are displayed. The value of LookupDisplayIndex must
  be less than the number of fields specified by the LookupField property.

----------------------------------------------------------------------------------------------------
@@TJvDBLookupEdit.LookupField
Summary
  Identifies the field in the LookupSource dataset that must match the value of the DataField field.
Description
  Use LookupField to link the LookupSource of the lookup control to the DataSource. Although the name
  of the field specified as the LookupField does not have to be the same as the name of the field
  specified as the DataField, the two fields must have the same values.
  
  After specifying the LookupField, choose which field the lookup control will actually display with
  the LookupField property.
  
  If the DataField field is a lookup field, don't specify anything for LookupField or LookupField;
  lookup controls automatically use the data field's LookupKeyFields property for LookupField.

----------------------------------------------------------------------------------------------------
@@TJvDBLookupEdit.LookupSource
Summary
  Identifies a data source for the data displayed in the lookup control.
Description
  Set LookupSource to the data source that contains the LookupField and LookupField fields. If the
  DataField field is a lookup field, don't specify anything for LookupSource; data controls
  automatically use the lookup field's LookupDataSet property to create a data source.

----------------------------------------------------------------------------------------------------
@@TJvDBLookupEdit.LookupValue
Summary
  Write here a summary (1 line)
Description
  Returns the look up value Run-time only Read-only

----------------------------------------------------------------------------------------------------
@@TJvDBLookupEdit.DropDownCount
Summary
  Specifies the number of items displayed in the drop-down list of the TJvDBLookupEdit.
Description
  Use DropDownRows to specify how many rows appear in the drop down list of the combo box. If there
  are more lookup items than DropDownRows, the lookup list box displays a scroll bar.
  
  The default value for DropDownCount is 7.

----------------------------------------------------------------------------------------------------
@@TJvDBLookupEdit
<TITLEIMG TJvDBLookupEdit>
#JVCLInfo
<GROUP JVCL.EditsMemosAndCombos.Combos,JVCL.DBAware.EditsMemosAndCombos.Combos>
<FLAG Component>
Summary
  Data aware edit with look up
Description
  Write here a description

----------------------------------------------------------------------------------------------------
@@TJvDBLookupCombo.TabSelects
Summary
  Write here a summary (1 line)
Description
  Write here a description
See Also
  List here other properties, methods (comma seperated)
  Remove the 'See Also' section if there are no references

----------------------------------------------------------------------------------------------------
@@TJvDBLookupCombo.Text
Summary
  Write here a summary (1 line)
Description
  Write here a description
See Also
  List here other properties, methods (comma seperated)
  Remove the 'See Also' section if there are no references

----------------------------------------------------------------------------------------------------
@@TJvDBLookupCombo.OnDropDown
Summary
  Occurs immediately before the lookup list is opened.
Description
  Write an OnDropDown event handler to take specific action before the lookup list is displayed to
  the user. The list can be opened by the user or by calling the DropDown method.
See Also
  TJvDBLookupCombo.DropDown, TJvDBLookupCombo.OnCloseUp

----------------------------------------------------------------------------------------------------
@@TJvDBLookupCombo.OnCloseUp
Summary
  Occurs immediately after an opened or �dropped-down� list is closed.
Description
  Write an OnCloseUp event handler to respond when the combo box list is closed. When the list is
  closed, the value that corresponds to the selected lookup value is assigned to the field. The list
  can be closed by the user or by calling the CloseUp method.
See Also
  TJvDBLookupCombo.CloseUp, TJvDBLookupCombo.OnDropDown

----------------------------------------------------------------------------------------------------
@@TJvDBLookupCombo.ListVisible
Summary
  Specifies whether the lookup list is open or �dropped-down�.
Description
  Read ListVisible to determine whether the list of lookup values is currently in the open
  (dropped-down) position. If ListVisible is true, the list is open; if IsDropDown is false, the list
  is closed.
See Also
  TJvDBLookupCombo.OnDropDown, TJvDBLookupCombo.IsDropDown

----------------------------------------------------------------------------------------------------
@@TJvDBLookupCombo.IsDropDown
Summary
  Specifies whether the lookup list is open or �dropped-down�.
Description
  IsDropDown returns the same value as ListVisible.
See Also
  TJvDBLookupCombo.ListVisible

----------------------------------------------------------------------------------------------------
@@TJvDBLookupCombo.EscapeKeyReset
Summary
  Determines whether the associated field is cleared. when the Escape key is pressed.
Description
  Set EscapeClear to let users enter a �blank� value into the associated field. When the user presses
  the Escape key, the associated field is cleared.

----------------------------------------------------------------------------------------------------
@@TJvDBLookupCombo.DropDownWidth
Summary
  Specifies the width, in pixels, of the drop-down list.
Description
  Use DropDownWidth to customize the width of the drop-down list. If DropDownWidth is 0 (the
  default), the drop-down list is the same width as the combo box.
  
  DropDownWidth is useful when displaying multiple lookup fields, and therefore, multiple columns in
  the database lookup combo box.
See Also
  TJvDBLookupCombo.DropDownAlign, TJvDBLookupCombo.DropDownCount, TJvDBLookupCombo.OnDropDown

----------------------------------------------------------------------------------------------------
@@TJvDBLookupCombo.DropDownCount
Summary
  Specifies the number of items displayed in the drop-down list of the TJvDBLookupCombo.
Description
  Use DropDownRows to specify how many rows appear in the drop down list of the combo box. If there
  are more lookup items than DropDownRows, the lookup list box displays a scroll bar.
  
  The default value for DropDownRows is 7.
See Also
  TJvDBLookupCombo.DropDownAlign, TJvDBLookupCombo.DropDownCount, TJvDBLookupCombo.OnDropDown

----------------------------------------------------------------------------------------------------
@@TJvDBLookupCombo.DropDownAlign
Summary
  Specifies how the drop-down list is aligned relative to its edit box.
Description
  Use DropDownAlign to specify the position of the drop-down list relative to the edit region of the
  lookup combo box. The drop down list can be aligned to the left or right, or centered with the edit
  box.
See Also
  TJvDBLookupCombo.DropDownCount, TJvDBLookupCombo.DropDownWidth, TJvDBLookupCombo.OnDropDown

----------------------------------------------------------------------------------------------------
@@TJvDBLookupCombo.DropDown
Summary
  Opens or �drops down� the lookup list so that the user can choose a lookup value.
Description
  Call DropDown to programmatically open the lookup list. Before the list is displayed, an OnDropDown
  event is generated.

----------------------------------------------------------------------------------------------------
@@TJvDBLookupCombo.DisplayAllFields
Summary
  Specifies whether to display the values of all lookup fields in the edit box.
Description
  When LookupDisplay is set to multiple fields, you can use DisplayAllFields to control whether all
  lookup fields are shown in the edit box, or only the value of LookupField.
See Also
  TJvDBLookupCombo.LookupDisplay

----------------------------------------------------------------------------------------------------
@@TJvDBLookupCombo.DisplayValues
Summary
  References the displayed values by their positions.
Description
  Use DisplayValues to read the value of a lookup field at a particular position, if ListStyle is
  lsFixed.
  
  If ListStyle is set to lsDelimited than DisplayValues always returns the value of DisplayValue.
  
  Index gives the position of the display value, where 0 is the position of the first display value,
  1 is the position of the second display value, and so on.
See Also
  TJvDBLookupCombo.ListStyle, TJvDBLookupCombo.LookupDisplay

----------------------------------------------------------------------------------------------------
@@TJvDBLookupCombo.DeleteKeyClear
Summary
  Write here a summary (1 line)
Description
  Write here a description
See Also
  List here other properties, methods (comma seperated)
  Remove the 'See Also' section if there are no references

----------------------------------------------------------------------------------------------------
@@TJvDBLookupCombo.CloseUp
Summary
  Closes an opened or �dropped-down� list.
Description
  Call CloseUp to programmatically close the list of the lookup combo box. The Accept parameter
  determines whether to modify the field value with the value that corresponds to the selected value
  in the lookup list.
Parameters
  Accept - Description for this parameter

----------------------------------------------------------------------------------------------------
@@TJvDBLookupCombo
<TITLEIMG TJvDBLookupCombo>
#JVCLInfo
<GROUP JVCL.EditsMemosAndCombos.Combos,JVCL.DBAware.EditsMemosAndCombos.Combos>
<FLAG Component>
Summary
  Expands the capabilities of a regular data aware combo box.
Description
  It allows you to enter mapped storage and display values so that you can display understandable
  text versions of stored codes in your table, instead of displaying only the codes themselves where
  users have to remember what they all mean.
  
  Alternatively you could use a TJvDBLookupCombo to display one field from a lookup source, and store
  a different field, but the TJvDBComboBox's drop-down list comes directly from a string list and not
  required the lookup source.

----------------------------------------------------------------------------------------------------
@@TGetImageEvent
<TITLE TGetImageEvent type>
Summary
  The type for an OnGetImage event handler.
Description
  Write here a description.
Parameters
  Sender     - Description for this parameter
  IsEmpty    - Description for this parameter
  Graphic    - Description for this parameter
  TextMargin - Description for this parameter
See Also
  TJvDBLookupEdit.OnGetImage, TJvLookupControl.OnGetImage

----------------------------------------------------------------------------------------------------
@@TJvLookupControl.EmptyItemColor
Summary
  The color to use for the empty row.
Description
  Use EmptyItemColor to specify the color to use for the empty row that is displayed in the drop-down
  list of the control.
  
  If DisplayEmpty is an empty string, this property has no effect.
See Also
  TJvLookupControl.DisplayEmpty

----------------------------------------------------------------------------------------------------
@@TJvLookupControl.EmptyStrIsNull
Summary
  Specifies whether to use Null as the empty value.
Description
  If EmptyValue is NOT an empty string, this property has no effect.
See Also
  TJvLookupControl.EmptyValue

----------------------------------------------------------------------------------------------------
@@TJvLookupControl.EmptyValue
Summary
  Specifies which lookup value is the empty value.
Description
  Use EmptyValue to use a specific string as the empty lookup value.
  
  Alternatively, set EmptyValue to an empty string and EmptyStrIsNull to true to specify Null as the
  empty value.
See Also
  TJvLookupControl.EmptyStrIsNull, TJvLookupControl.DisplayEmpty

----------------------------------------------------------------------------------------------------
@@TJvLookupControl.Field
Summary
  Identifies the TField object the lookup control represents.
Description
  Use the Field object to directly access the field component for the field the lookup control
  represents. If Field is a lookup field, the properties of the field component describe the
  relationship between the field and the lookup dataset. To allow a lookup control to represent a set
  of lookup values that do not come from another dataset, use the DataField property to bind to a
  lookup field and use the LookupList property of the TField object to specify a list of lookup
  values.

----------------------------------------------------------------------------------------------------
@@TJvLookupControl.FieldsDelimiter
Summary
  Specifies the character used to delimit the field values in the drop-down lookup list.
Description
  Use FieldsDelimiter to display another character as field delimiter than the default comma
  character.
  
  FieldsDelimiter is only used if ListStyle is set to lsDelimited. Also LookupFormat must have no
  value.
See Also
  TJvLookupControl.ListStyle, TJvLookupControl.LookupFormat

----------------------------------------------------------------------------------------------------
@@TJvLookupControl.IgnoreCase
Summary
  Controls whether incremental searching is done in a case-sensitive or case-insensitive manner.
Description
  If IgnoreCase property is True (default), the incremental search through the lookup list is done
  without regard to case in the dataset's data.
See Also
  TJvLookupControl.LookupDisplay, TJvLookupControl.LookupDisplayIndex

----------------------------------------------------------------------------------------------------
@@TJvLookupControl.IndexSwitch
Summary
  Controls whether incremental searching is done using available indices or not.
Description
  If IndexSwitch property is True (default) and a TTable component is linked to the LookupSource
  property, the incrementally search will use available indices of the lookup table.
See Also
  TJvDBLocate

----------------------------------------------------------------------------------------------------
@@TJvLookupControl.ItemHeight
Summary
  Specifies the height, in pixels, of the items in the drop-down list.
Description
  Use ItemHeight to specify the height needed to draw the items in the list.

----------------------------------------------------------------------------------------------------
@@TJvLookupControl.KeyValue
Summary
  Represents the common value of the LookupField field and the DataField field.
Description
  Use LookupField to determine the value represented by the lookup control (not the value displayed
  by the lookup control). When KeyValue is set, the lookup control attempts to find a record from the
  LookupSource's dataset where the value of LookupField matches KeyValue. If such a match is found,
  the
  lookup control displays the value of LookupField on that record.
See Also
  TJvLookupControl.DataField, TJvLookupControl.LookupField, TJvLookupControl.IgnoreCase,
  TJvLookupControl.IndexSwitch

----------------------------------------------------------------------------------------------------
@@TJvLookupControl.ListStyle
Summary
  Determines how the lookup list displays its items.
Description
  ListStyle determines how the lookup list displays its items when multiple fields specified by the
  LookupDisplay property.
  
  By default, style is lsFixed, meaning that the each display field always takes up a fixed width
  accordingly to their DisplayWidth properties. The lsDelimited style meaning that each field takes a
  variable width, and the fields are separated by the FieldsDelimiter character.
See Also
  TJvLookupControl.FieldsDelimiter

----------------------------------------------------------------------------------------------------
@@TJvLookupControl.Locate
Summary
  Locates a field.
Description
  Call Locate to locate parameter AValue in the lookup dataset for field SearchField.
  
  By changing the value of property IgnoreCase you can control the way the case influences the result.
Parameters
  SearchField - Field name on which to search.
  AValue      - Value to match in the search field.
  Exact       - If Exact is true, the function locates only records where SearchField is fully equal 
                to AValue, otherwise the function operates a partial match.
See Also
  TJvLookupControl.IgnoreCase

----------------------------------------------------------------------------------------------------
@@TJvLookupControl.LookupDisplay
Summary
  Identifies the field or fields whose values are displayed in the lookup list.
Description
  The lookup control represents a field in one table by displaying the value of one or more
  corresponding fields in another table. LookupField is the name of the field or fields in the lookup
  table that are actually displayed, as opposed to the DataField, which is the field in the
  DataSource that the lookup control actually represents, or the LookupField, which is the field in
  the lookup table with the same value as the DataField.
  
  To fully specify the list fields, both a dataset for the lookup table and the fields within that
  dataset must be defined. The LookupSource property of the lookup control specifies the dataset for
  the lookup table.
  
  LookupField can represent more than one field. Separate multiple field names with semicolons.
  
  Before specifying LookupField, specify the link between the two datasets using the LookupField
  property. If LookupField is not set, lookup controls display LookupField field values by default.
  If the DataField field is a lookup field, don't specify anything for LookupField; the data controls
  automatically use the lookup field's LookupResultField property as LookupField.

----------------------------------------------------------------------------------------------------
@@TJvLookupControl.LookupDisplayIndex
Summary
  Specifies which field from the LookupField property is used for incremental searching.
Description
  When the LookupField property specifies more than one field, use LookupDisplayIndex to specify
  which of those fields is the one to use for incremental searches. For the TDBLookupComboBox object,
  LookupDisplayIndex also determines which field appears in the edit region of the combo box.
  
  LookupDisplayIndex allows the most important value to appear in a position other than the first when all the values of the LookupField fields are displayed. The value o
   LookupDisplayIndex must be less than the number of fields specified by the LookupField property.
  -- When multiple fields specified by the LookupDisplay property, the LookupDisplayIndex specifies
  index of a field in the LookupDisplay list which will be use in the DisplayValue property and will
  be display in the editor of the
  TJvDBLookupCombo component.

----------------------------------------------------------------------------------------------------
@@TJvLookupControl.LookupField
Summary
  Identifies the field in the LookupSource dataset that must match the value of the DataField field.
Description
  Use LookupField to link the LookupSource of the lookup control to the DataSource. Although the name
  of the field specified as the LookupField does not have to be the same as the name of the field
  specified as the DataField, the two fields must have the same values.
  
  After specifying the LookupField, choose which field the lookup control will actually display with
  the LookupField property.
  
  If the DataField field is a lookup field, don't specify anything for LookupField or LookupField;
  lookup controls automatically use the data field's LookupKeyFields property for LookupField.

----------------------------------------------------------------------------------------------------
@@TJvLookupControl.LookupFormat
Summary
  Specifies the format to use to display the field values in the lookup list.
Description
  Use LookupFormat to display the lookup field values in a specific format in the lookup list. The
  format string must have the same number of format specifiers as the number of fields (specified by
  LookupDisplay). The only allowed format specifier is "%s".
  
  For example, you have 3 fields in the lookup table: FIRSTNAME, LASTNAME and AGE. If these fields
  have the values 'John', 'Doe', 25 resp. then you want it to be displayed as 'Doe, John (25)'. In
  this case you have to set the properties as follows:
  <CODE>
  ListStyle := lsDelimited;
  LookupDisplay := 'LASTNAME;FIRSTNAME;AGE';
  LookupFormat := '%s, %s (%s)'; </CODE>
  Note
  Property LookupFormat is ignored if ListStyle is set to lsFixed.
See Also
  TJvLookupControl.ListStyle

----------------------------------------------------------------------------------------------------
@@TJvLookupControl.LookupSource
Summary
  Identifies a data source for the data displayed in the lookup control.
Description
  Set LookupSource to the data source that contains the LookupField and LookupDisplay fields. If the
  DataField field is a lookup field, don't specify anything for LookupSource; data controls
  automatically use the lookup field's LookupDataSet property to create a data source.

----------------------------------------------------------------------------------------------------
@@TJvLookupControl.OnChange
Summary
  Write here a summary (1 line)
Description
  Write here a description
See Also
  List here other properties, methods (comma seperated)
  Remove the 'See Also' section if there are no references

----------------------------------------------------------------------------------------------------
@@TJvLookupControl.OnGetImage
Summary
  Occurs when an image is requested.
Description
  Write an OnGetImage event handler to supply an image before they are used.

----------------------------------------------------------------------------------------------------
@@TJvLookupControl.ReadOnly
Summary
  Determines if the user can change the value of the field.
Description
  Set ReadOnly to specify whether the control should be used for display purposes only. If ReadOnly
  is true, the control can only be used to display the value of the field on the current record. If
  ReadOnly is false, the user can use the control to edit the field's value.

----------------------------------------------------------------------------------------------------
@@TJvLookupControl.ResetField
Summary
  Write here a summary (1 line)
Description
  Forces the value to EmptyValue and generates an internal Click

----------------------------------------------------------------------------------------------------
@@TJvLookupControl.UseRecordCount
Summary
  Specifies whether the control is allowed to use RecordCount.
Description
  Use the UseRecordCount property to allow or forbid the control to use the RecordCount property of
  the lookup dataset. Record counting can be a costly operation, especially for SQL queries that
  return large result sets.
  
  If UseRecordCount is set to false then the position of the scrollbar's thumb of the lookup list is
  only an approximation.

----------------------------------------------------------------------------------------------------
@@TJvLookupControl.Value
Summary
  Write here a summary (1 line)
Description
  Write here a description
See Also
  List here other properties, methods (comma seperated)
  Remove the 'See Also' section if there are no references

----------------------------------------------------------------------------------------------------
@@TLookupListStyle
<TITLE TLookupListStyle type>
Summary
  Enumerates lookup list styles.
Description
  Use the TLookupListStyle type to specify a lookup list style when displaying multiple fields.
See Also
  TJvLookupControl.ListStyle, TJvDBLookupEdit.ListStyle

----------------------------------------------------------------------------------------------------
@@TLookupListStyle.lsFixed
Each display field takes up a fixed width accordingly to their DisplayWidth properties.

----------------------------------------------------------------------------------------------------
@@TLookupListStyle.lsDelimited
Each field takes a variable width, and the fields are separated by a specific character.

----------------------------------------------------------------------------------------------------
@@TLookupSourceLink
Summary
  Write here a summary (1 line)
Description
  Write here a description

