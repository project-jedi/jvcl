@@$JVCL.Info.Plugins
<GROUP $JVCL.Info>
<TITLE Writing plugins and plugin enabled applications>
  In order to allow developers to extend applications a mechanism is needed to be able to
  communicate between the main application and the extensions. These extensions are commonly
  refered to as plugins as this is a pretty accurate description of what it does: an end user
  places an extension's DLL (or a runtime package) in the applications folder and the extension
  will be available in the main application.

  A good example of an application that has plugin capabilities is WinAMP. Any developer can write
  an extension for WinAMP. WinAMP supports various types of plugins: User Interface plugins, input
  plugins, output plugins, sound processing plugins and more. All an end user needs to do is make
  sure the plugin is an the appropiate folder and WinAMP will load it.

  Extensions commonly provide some additional functionality which will have to be made available in
  the main application (referred to as the host application). This is usually done by adding an
  option to a menu or a button to a toolbar. Some extensions will not provide user selectable
  functionality, but rather provide some form of processing. These plugins will therefor not provide
  additional menu items or buttons, but will be linked into a some kind of processing chain.

  The plugin capabilities provided by the JEDI-VCL will service both types of extensions. Each
  plugin may support either commands, message processing or both. A plugin is not limited to a
  single command or a single message.

  Click <LINK $JVCL.Info.Plugins.PluginWriting, here> for more information regarding writing
  plugins. Click <LINK $JVCL.Info.Plugins.HostWriting, here> for more information regarding plugin
  enabling applications.
----------------------------------------------------------------------------------------------------
@@$JVCL.Info.Plugins.PluginWriting
<GROUP $JVCL.Info.Plugins>
<TITLE Writing plugins>
  Writing plugins requires some knowledge about the host application. More precisely, you need to
  know if an application allows commands to be added to it's menu and/or toolbar if and which
  messages it can send and if it allows plugins to be configured and how it wants to do this.

  Plugins can also be written to provide additional MDI children, additional forms, data modules
  and probably a lot more. It all more or less depends on what the host application supports.

  In any case, it's vital to study the host applications documentation to be able to write plugins
  the host application can actually cope with.
  
  This chapter explains some of the basics behind writing plugins and provides a tutorial for
  writing a simple plugin.
----------------------------------------------------------------------------------------------------
@@JVCL.Info.Plugins.PluginWriting.Basics
<GROUP $JVCL.Info.Plugins.PluginWriting>
<TITLE Writing plugins: the basics>
  To create plugin for an application you generally use the provided JEDI Plugin Wizard, which can
  be accessed by going to File|New... in Delphi. Head over to the Projects tab, select the JEDI
  Plugin Wizard icon and press OK.
  
  A dialog is shown where you can enter the plugin's name and type of plugin.
  
  The name denotes the name of the project (prefixed with <B>Plg</B> as well as the name of the
  unit (prefixed with <B>Plugin</B>) and the TJvPlugin instance (prefixed with <B>Tplg</B>) that
  will be created.
  
  The plugin type determines if a new DLL project or a new Package project will be created for the
  plugin.
  
  After pressing OK, a new project will be created, containing a unit with the TJvPlugin instance.
  
  To complete the plugin you can follow these tasks:
    * Specify the plugin ID. This ID must be unique in the host application, so it is wise to use
      a name that includes your company or your own name. Alternatively, you could assign a GUID
      (<B>G</B>lobal <B>U</B>nique <B>ID</B>enditifier), but that would make the ID useless in the
      host application user interface as means of identifying which plugins are loaded. Internally,
      the host application doesn't care what this property contains, as long as it's unique.
    * Optionally specify the author, copyright, description and version of your plugin.
    * Optionally add one or more commands to the Commands collection of the TJvPlugin instance,
      providing a caption and an OnExecute handler and optionally a hint, custom data and/or a
      bitmap for each command.
    * Optionally assign a handler to the OnInitialize event.
    * Optionally assign a handler to the OnConfigure event.
    * Optionally assign a handler to the OnPluginMessage event
  All tasks (except the first) are optional, but a useful plugin requires either an OnPluginMessage
  handler or a least one command to be assigned. You should check the host applications (technical)
  documentation to find out what the host application expects from the plugin.
----------------------------------------------------------------------------------------------------
@@$JVCL.Info.Plugins.PluginWriting.Tutorials
<GROUP $JVCL.Info.Plugins.PluginWriting>
<TITLE Writing plugins: tutorials>
  This chapter contains a number of tutorials. Most tutorials just expand on the basic tutorial.
----------------------------------------------------------------------------------------------------
@@JVCL.Info.Plugins.PluginWriting.Tut01
<GROUP $JVCL.Info.Plugins.PluginWriting.Tutorials>
<TITLE Writing plugins: Tutorial 1: Basic tutorial>
  This tutoral will create a simple plugin with one command in it (a simple 'Hello world!') example.
  
  We start this tutorial by creating a new plugin project. Go to File|New..., select the
  <B>Projects</B> tab, select then <B>JEDI Plugin Wizard</B> and press OK. In the dialog that
  appears, type in the name of the plugin, in this case we'll take enter "Tutorial1". In the box
  below we'll select <B>Package-Plugin</B> as the plugin type (this is the default).
  
  Pressing OK will create a new project <B>PlgTutorial1.dpk</B> and a unit called
  <B>PluginTutorial1.pas</B>. The unit contains a data module named <B>Tutorial1</B>. When you
  select that module, the object inspector will show some additional properties and events.
  
  First thing to do is to specify the ID of the plugin for easy identification by the host
  application. Set PluginID property to <B>JEDI.Tutorial1</B>. By including the <B>JEDI.</B> part,
  we make sure that this tutorial is unique in the host application, even if another company has
  written a plugin with the same name.
  
  To make the plugin selection interface in the host application more usefull, we can now assign
  proper values to the following properties:
  <TABLE>
    Property            Value
    ========            =====
    Author              Your name, your companies name or both e.g. <B>John Doe</B>
    Copyright           The copyright notice for your plugin, e.g. <B>Copyright � 2003 Project JEDI</B>
    Description         A short description of the plugin, e.g. <B>My first plugin</B>
    PluginVersion       The version of your plugin, e.g. <B>1.0</B> or <B>1.0 RC1</B>.
  </TABLE>
  
  Our tutorial plugin will provide one command that will just show a message box with the text 
  'Hello world!' (programmers are so creative when they write tutorials and examples ;) ).
  
  To create the command will doubleclick on the Commands property to show a familiar editor. Press
  <B>Insert</B> to create a new command. The object inspector will now show the properties and
  events for our command. To keep the tutorial simple, we'll set the <B>Name</B> property to
  'Command1' and the <B>Caption</B> property to 'Test1'.
  
  Next, doubleclick on the <B>OnExecute</B> event to generate a new event. In this event we'll
  show our most important message. The code will look like this:
    
<CODE>
<B>procedure</B> TTutorial1.Tutorial1Commands0Execute(Sender: TObject);
<B>begin</B>
  ShowMessage('Hello world!');
<B>end;</B>
</CODE>

  Compile the package. A message will popup saying that the JVCL runtime package should be added to
  the requires node. If the host application specifies it uses the JVCL runtime package, you should
  add it, otherwise you can ignore the message and the warning that follows.
  
  At this point your first plugin is ready. Make sure the plugin .bpl file is placed in the same
  folder as your test host, and run the test host. A suitable test host can be build by following
  the host application tutorial.
## !!! TODO: link to the host application tutorial.
  If all is well, you should now see a new menu item in your test host and when you click on it,
  you'll see our message.
----------------------------------------------------------------------------------------------------
@@$JVCL.Info.Plugins.HostWriting
<GROUP $JVCL.Info.Plugins>
<TITLE Writing host applications>
  Writing plugin enabled applications, or hosts, requires some thinking ahead. You should determine
  upfront which types of plugins you want to support.

  Are plugins allowed to add commands to your menu or toolbar? If so, how should they specify the
  location? How is the host to determine if the command is actually enabled?

  Should plugins be able to receive generic messages (notifications)? If so, what types of
  notifications should there be and is there any additional data required?

  Can plugins provide MDI children, forms or data modules? Can they provide other functionality? Is
  there a configuration possibility and how to implement it?

  It's best to fully define the possibilities of your plugins before you even start coding or
  changing your application, or you'll soon be lost.
