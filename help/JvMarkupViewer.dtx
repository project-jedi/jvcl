##Package: Jans
##Status: Needs review
----------------------------------------------------------------------------------------------------
@@JvMarkupViewer.pas
Summary
    Contains the TJvMarkupViewer component.
Author
  Jan Verhoeven
----------------------------------------------------------------------------------------------------
@@TJvMarkupViewer.Text
Summary
  HTML markup to render as the content of the control.
Description
  Text is a TCaption property that specifies the HTML markup rendered as the content for the
  HTML-enabled control. Values in the Text property are parsed into the HTML elements need to draw
  the contents of the control using the Paint method. Changing the value in either the Text or Font
  properties causes the control to be redrawn.
  
  Text can contain plain text and any of the following HTML tags:
  
  <TABLE>
    Tag                  Description
    -------------------  ---------------------------------------------------------------------
    \<br\>               Break tag. Performs a newline or carriage return when rendering the
                           HTML tag.
    \<b\>\<\/b\>         Bold tag. Renders using the Bold attribute for the current font.
    \<i\>\<\/i\>         Italic tag. Renders using the Italic attribute for the current font.
    \<u\>\<\/u\>         Underline tag. Renders using the Underline attribute for the current
                           font.
    \<font\>\<\/font\>   Font tag. Allows defining the font used to render the values between
                           the start and end tags. Use the face, size, and color attributes
                           to specify the values to use for the font element.
  </TABLE>
  
  The following attributes are recognized for HTML markup in the Text property:
  
  <TABLE>
    Attribute  Description
    ---------  -------------------------------------------------------------------------------
    face       Font name to use for the HTML element. Must match the name for a font
                 installed on the computer.
    size       Font size for the HTML element. The unit measure for size is expressed in
                 points (1/72 inch).
    color      Font color for the HTML element. Font may be expressed as a decimal value
                 (using #nnnnnn) or a color name (Red, Green, Blue, Lime, Maroon, etc).
  </TABLE>
  
  Attributes attached to an HTML element must be enclosed in double quote characters ("). For
  instance:
  
  <CODE>
  <font face="Arial" size="14" color="Navy">Title</font><br> </CODE>
  
  Use the MarginLeft, MarginRight, and MarginTop properties to control the spacing needed prior to
  the rendered HTML content for the control using the Alignment property.
  
  Use BackColor to specify the default background color for HTML elements rendered in the control.
See Also
  TJvHTMLElementStack, TJvHTMLElement, TJvMarkupViewer.MarginLeft, TJvMarkupViewer.MarginRight,
  TJvMarkupViewer.MarginTop, TJvMarkupViewer.BackColor

----------------------------------------------------------------------------------------------------
@@TJvMarkupViewer.MarginTop
Summary
  Specifies the top margin for content rendered in the HTML-enabled control.
Description
  MarginTop is an Integer property that specifies the top margin in pixels for content rendered in the
  HTML-enabled control. The default value for MarginTop is 5.
  
  MarginLeft, MarginRight, and MarginTop are used when rendering the HTML markup in the Text for the
  control using the Paint method. MarginLeft, MarginRight, and MarginTop determine the edge
  boundaries for the Alignment and Align properties within the client area for the control.
  
  Changing the value in the MarginLeft, MarginRight, or MarginTop properties causes the control to be
  redrawn.
See Also
  TJvMarkupViewer.MarginLeft, TJvMarkupViewer.MarginRight, TJvMarkupViewer.Text, TJvMarkupViewer.Paint

----------------------------------------------------------------------------------------------------
@@TJvMarkupViewer.Paint
Summary
  Renders the HTML markup for the HTML-enabled control.
Description
  Paint is an overridden method in TJvMarkupViewer that implements rendering of the HTML markup in
  Text for the HTML-enabled control.
  
  Rendering uses the Canvas for the control to recalculate the position for HTML elements found in the
  Text property. HTML element positions are relative to the values specified in the MarginLeft,
  MarginRight, and MarginTop properties. The value in MarginBottom is also added to the overall total
  height of the rendered control. The right boundary for the control is adjusted to allow space for
  drawing the ScrollBar for the control.
  
  BackColor is used to set the background and transparency colors for the rendered HTML elements.
  
  To provide scrollability, drawing HTML elements is performed on the Canvas for a TBitmap instance.
  Changes to the position of the ScrollBar cause the visible portion of the bitmap to be repositioned
  in the client area for the control.
  
  Internally, Paint uses TJvHTMLElementStack and TJvHTMLElement to calculate the flow and alignment
  of the HTML elements in the client area of the control. This includes wrapping of HTML elements
  that cannot be draw in their entirety on the current line.
See Also
  TJvMarkupViewer.Text, TJvMarkupViewer.MarginLeft, TJvMarkupViewer.MarginRight,
  TJvMarkupViewer.MarginTop, TJvMarkupViewer.BackColor, TJvHTMLElementStack, TJvHTMLElement

----------------------------------------------------------------------------------------------------
@@TJvMarkupViewer.BackColor
Summary
  Background color used when rendering the HTML markup for the control.
Description
  BackColor is a TColor property that specifies the background color used when rendering the HTML
  markup in the Text property. BackColor is used when Paint is called to render the HTML markup. The
  value in BackColor is assigned as both the Brush and Transparency colors for the Canvas used to
  draw the control.
  
  The default value for BackColor is clWhite, as assigned in the constructor for the component.
  Changing the value in BackColor or Text causes the control to be redrawn.
See Also
  TJvMarkupViewer.Paint, TJvMarkupViewer.Text

----------------------------------------------------------------------------------------------------
@@TJvMarkupViewer.MarginLeft
Summary
  Specifies the left-hand margin for content rendered in the HTML-enabled control.
Description
  MarginLeft is an Integer property that specifies the left-hand margin in pixels for content
  rendered in the HTML-enabled control. The default value for MarginLeft is 5.
  
  MarginLeft, MarginRight, and MarginTop are used when rendering the HTML markup in the Text for the
  control using the Paint method. MarginLeft, MarginRight, and MarginTop determine the edge
  boundaries for the Alignment and Align properties within the client area for the control.
  
  Changing the value in the MarginLeft, MarginRight, or MarginTop properties causes the control to be
  redrawn.
See Also
  TJvMarkupViewer.MarginRight, TJvMarkupViewer.MarginTop, TJvMarkupViewer.Text, TJvMarkupViewer.Paint

----------------------------------------------------------------------------------------------------
@@TJvMarkupViewer.MarginRight
Summary
  Specifies the right-hand margin for content rendered in the HTML-enabled control.
Description
  MarginRight is an Integer property that specifies the right-hand margin in pixels for content
  rendered in the HTML-enabled control. The default value for MarginRight is 5.
  
  MarginLeft, MarginRight, and MarginTop are used when rendering the HTML markup in the Text for the
  control using the Paint method. MarginLeft, MarginRight, and MarginTop determine the edge
  boundaries for the Alignment and Align properties within the client area for the control.
  
  Changing the value in the MarginLeft, MarginRight, or MarginTop properties causes the control to be
  redrawn.
See Also
  TJvMarkupViewer.MarginLeft, TJvMarkupViewer.MarginTop, TJvMarkupViewer.Text, TJvMarkupViewer.Paint

----------------------------------------------------------------------------------------------------
@@TJvMarkupViewer
<TITLEIMG TJvMarkupViewer>
#JVCLInfo
<GROUP JVCL.??>
<FLAG Component>
Summary
  Implements a scrollable HTML-enabled viewer component.
Description
  TJvMarkupViewer is a TJvCustomControl descendant that implements a scrollable HTML-enabled viewer
  component.
  
  TJvMarkupViewer provides the ability to parse and render HTML markup assigned to the Text property
  for the control. TJvMarkupViewer uses a TBitmap instance to render the contents of the HTML markup
  in the Text property, and to allow scrolling of the rendered content in the visible bounds of the
  component.
  
  TJvMarkupViewer recognizes the following HTML tags in the Text property:
  
  <TABLE>
    Tag                  Description
    -------------------  ---------------------------------------------------------------------
    \<br\>               Break tag. Performs a newline or carriage return when rendering the
                           HTML tag.
    \<b\>\<\/b\>         Bold tag. Renders using the Bold attribute for the current font.
    \<i\>\<\/i\>         Italic tag. Renders using the Italic attribute for the current font.
    \<u\>\<\/u\>         Underline tag. Renders using the Underline attribute for the current
                           font.
    \<font\>\<\/font\>   Font tag. Allows defining the font used to render the values between
                           the start and end tags. Use the face, size, and color attributes
                           to specify the values to use for the font element.
  </TABLE>
  
  The following attributes are recognized for HTML markup in the Text property:
  
  <TABLE>
    Attribute  Description
    ---------  -------------------------------------------------------------------------------
    face       Font name to use for the HTML element. Must match the name for a font
                 installed on the computer.
    size       Font size for the HTML element. The unit measure for size is expressed in
                 points (1/72 inch).
    color      Font color for the HTML element. Font may be expressed as a decimal value
                 (using #nnnnnn) or a color name (Red, Green, Blue, Lime, Maroon, etc).
  </TABLE>
  
  Attributes attached to an HTML element must be enclosed in double quote characters ("). For
  instance:
  
  <CODE>
  <font face="Arial" size="14" color="Navy">Title</font><br> </CODE>
  
  TJvMarkupViewer uses internal instances of TJvHTMLElementStack during parsing and rendering of HTML
  tags assigned to the Text property. TJvMarkupViewer implements overridden methods to handle Font and
  Text assignments to properties in the control, and to perform painting of the rendered control.
  
  Use the MarginLeft, MarginRight, and MarginTop properties to control the spacing needed prior to
  the rendered HTML content for the control.
  
  Use TJvMarkupLabel to render HTML markup in a Label component.
See Also
  TJvCustomControl, TJvMarkupLabel, TJvHTMLElementStack, TJvHTMLElement

